
%    \iffalse
%
% pdfpages.dtx Copyright (C) 2001 Andreas MATTHIAS
%
% This program may be distributed and/or modified under the
% conditions of the LaTeX Project Public License, either version 1.2
% of this license or (at your option) any later version.
% The latest version of this license is in
%   http://www.latex-project.org/lppl.txt
% and version 1.2 or later is part of all distributions of LaTeX 
% version 1999/12/01 or later.
%
% Please send error reports and suggestions for improvements to
%   Andreas MATTHIAS <amat@kabsi.at>.
%
%<*!package>
% ===================================================================
%  @LaTeX-package-file{
%    author         = {Andreas MATTHIAS},
%    version        = "0.2e",
%    date           = "27. January 2002",
%    filename       = "pdfpages.sty",
%    address        = "",
%    telephone      = "",
%    email          = "amat@kabsi.at",
%    codetable      = "ISO/ASCII",
%    keywords       = "pdfLaTeX, PDF, include, insert, page",
%    dependences    = "ifthen, graphicx, eso-pic",
%    supported      = "yes",
%    docstring      = "Package for inserting pages of external
%                      PDF documents. (Requiring pdfLaTeX.)"
%  }
% ===================================================================
%</!package>
%
%<*dtx>
          \ProvidesFile{pdfpages.dtx}
%</dtx>
%    \fi
%
%    \CheckSum{1776}
%%
%% \CharacterTable
%%  {Upper-case    \A\B\C\D\E\F\G\H\I\J\K\L\M\N\O\P\Q\R\S\T\U\V\W\X\Y\Z
%%   Lower-case    \a\b\c\d\e\f\g\h\i\j\k\l\m\n\o\p\q\r\s\t\u\v\w\x\y\z
%%   Digits        \0\1\2\3\4\5\6\7\8\9
%%   Exclamation   \!     Double quote  \"     Hash (number) \#
%%   Dollar        \$     Percent       \%     Ampersand     \&
%%   Acute accent  \'     Left paren    \(     Right paren   \)
%%   Asterisk      \*     Plus          \+     Comma         \,
%%   Minus         \-     Point         \.     Solidus       \/
%%   Colon         \:     Semicolon     \;     Less than     \<
%%   Equals        \=     Greater than  \>     Question mark \?
%%   Commercial at \@     Left bracket  \[     Backslash     \\
%%   Right bracket \]     Circumflex    \^     Underscore    \_
%%   Grave accent  \`     Left brace    \{     Vertical bar  \|
%%   Right brace   \}     Tilde         \~}
%%
%    \iffalse
%<package>\NeedsTeXFormat{LaTeX2e}
%<package>\ProvidesPackage{pdfpages}[2002/01/27 v0.2e Insert pages of external PDF documents (AM)]
%    \fi
%         \ProvidesFile{pdfpages.dtx}[2002/01/27 v0.2e Insert pages of external PDF documents (AM)]
%    \iffalse
%
%<*driver>
\documentclass{ltxdoc}
\OnlyDescription
\usepackage{array}
  \newcommand*{\versionlist}{%
    \setlength{\topsep}{0pt}%
    \setlength{\itemsep}{0pt}%
    \setlength{\leftmargin}{20pt}%
    \setlength{\parsep}{0pt}}
  \newcommand{\marginstyle}
    {\raggedleft\itshape\small}
\begin{document}
  \DocInput{pdfpages.dtx}
\end{document}
%</driver>
%    \fi
%
%  \GetFileInfo{pdfpages.dtx}
%
%
%    \title{The \textsf{pdfpages} Package\thanks{This file
%           has version number \fileversion, last revised \filedate.}}
%    \author{Andreas MATTHIAS\\ \texttt{amat@kabsi.at}} 
%
%
%    \date{\filedate}
%
%    \maketitle
%
%    \begin{abstract}
%       \noindent
%       This package makes it easy to insert pages of external
%       PDF documents. It is based on PDF\LaTeX\ and does 
%       \textit{not} work with \LaTeX.
%    \end{abstract}
%
%    \tableofcontents
%
%
%    \section{Introduction}
%    Creating PDF documents it is sometimes useful to insert
%    pages of other, external PDF documents. This can be done
%    with the |\includegraphics| command from the
%    \texttt{graphics} package. But a simple
%    |\includegraphics{doc.pdf}| normally produces
%    `|Overfull \hbox|' and `|Overfull \vbox|' warnings,
%    because the size of the inserted pages does not match the
%    print space.
%    
%    The \texttt{pdfpages} package makes it easy to insert 
%    pages of external PDF documents without worrying about
%    the print space. It even gives you the possibility
%    to arrange several logical pages onto each sheet
%    of paper. (Like `psnup'.)
%    
%    \section{Usage}
%    \subsection{Package Options}
%    |\usepackage[|\meta{options}|]{pdfpages}|
%    
%    \begin{quote}
%    \begin{tabular}{@{}r@{~}r@{~}>{\raggedright}p{.7\linewidth}}
%        \meta{option} --  & |final|:& Inserts pages. This is the default.
%                                      \tabularnewline[.25\baselineskip]
%                          & |draft|:& Does not insert pages, but
%                                      prints a box and the filename
%                                      instead.
%    \end{tabular}
%    \end{quote}
%    
%    
%    \subsection{Commands}
%    
%    \DescribeMacro{\includepdf}
%    Inserts pages of an external PDF document.
%    
%    \medskip\noindent
%    |\includepdf[|\meta{key=val}|]{|\meta{filename}|}|
%    
%    \begin{quote}
%    \begin{tabular}{@{}l@{\,--~}>{\raggedright}p{.8\linewidth}}
%        \meta{key=val}   & A comma separated list of options
%                   using the \meta{key}=\meta{value} syntax.
%                   \tabularnewline
%        \meta{filename} & Filename of the PDF document.
%    \end{tabular}
%    \end{quote}
%
%    
%    \noindent
%    The following list describes all possible options of
%    |\includepdf|. All options are using the \meta{key=value} syntax.
%    If options are marked with `\emph{changed}' on the left margin, their
%    syntax or meaning has changed since pdfpages v0.1i. New options
%    are marked with `\emph{new}'.
%    
%    \newcommand*{\df}[1]{(Default: \mbox{|#1|})}
%
%    \setlength{\leftmargini}{\parindent}
%    \begin{itemize}
%
%    \item Main options:
%
%      \begin{description}
%      \makeatletter\@beginparpenalty=10000\makeatother
%        \item[pages] Selects pages to insert. The argument is a comma
%                     separated list containing page numbers 
%                     (|pages={3,5,6,8}|), ranges of page numbers 
%                     (|pages={4-9}|) or any combination.
%                     To insert empty pages use |{}|.
%                     \par
%                     E.\,g.: |pages={3,{},8-11,15}| will insert
%                     page 3, an empty page, and pages 8, 9, 10, 11,
%                     and 15.
%                     \par
%                     Page ranges are specified by the following syntax:
%                     \meta{m}|-|\meta{n}. This selects all pages between
%                     \meta{m} and \meta{n}. Omitting \meta{m} defaults
%                     to the first page, omitting \meta{n} defaults to
%                     the last page of the document.
%                     \par
%                     E.\,g.: |pages=-| will insert \emph{all} pages
%                     of the document.
%                     \par\df{pages=1}
%        \item[nup] \hskip0pt\marginpar{\marginstyle changed}
%                     Puts multiple logical pages onto each sheet
%                     of paper. The syntax of this option is: 
%                     |nup=|\meta{xnup}|x|\meta{ynup}. Where
%                     \meta{xnup} and \meta{ynup} specify the number of
%                     logical pages in vertical and horizontal direction,
%                     which are arranged on each sheet of paper.
%                     \df{nup=1x1}
%        \item[landscape] \hskip0pt\marginpar{\marginstyle changed}
%                     Specifies the format of the sheet
%                     of paper (\textit{not} of the logical pages).
%                     Either `true' or `false' (or no value, which
%                     is equivalent to `true'). \df{landscape=false}
%      \end{description}
%
%    \item Layout options:
%
%      \begin{description}
%      \makeatletter\@beginparpenalty=10000\makeatother
%        \item[delta] \hskip0pt\marginpar{\marginstyle new}
%                     Puts some horizontal and vertical space between the
%                     logical pages. The argument should be two dimensions,
%                     separated by space. See chapter~\ref{sec:layout} and
%                     figure~\ref{fig:layout}. \df{delta=0 0}.
%        \item[offset] Displaces origin of the inserted pages.
%                      The argument should be two dimensions, separated
%                      by space. See chapter~\ref{sec:layout} and 
%                      figure~\ref{fig:layout}. \df{offset=0 0}
%        \item[frame] Puts a frame around each logical page. Either `true'
%                     or `false' (or no value, which is equivalent to `true').
%                     \df{frame=false}
%        \item[column] \hskip0pt\marginpar{\marginstyle new}
%                     Pdfpages normally uses `row-major' layout, where 
%                     successive pages are placed in rows along the paper.
%                     The |column| option changes the output into a
%                     `column-major' layout, where successive pages are 
%                     arranged in columns down the paper.
%                     Either `true' or `false' (or no value, which
%                     is equivalent to `true'). \df{column=false}
%        \item[columnstrict] \hskip0pt\marginpar{\marginstyle new}
%                     By default the last page is not set in a
%                     strict `column-major' layout, if the logical
%                     page does not fill up the whole page. The
%                     |columnstrict| option forces a strict
%                     `column-major' layout for the last page.
%                     Either `true' or `false' (or no value, which
%                     is equivalent to `true'). \df{columnstrict=false}
%                     \par\hskip15pt
%                     \begin{minipage}[b]{110pt}
%                       \centering
%                       \begin{tabular}{|c|c|c|}\hline
%                       1 & 4 & \phantom{6}\\\hline
%                       2 & 5 &\\\hline
%                       3 &&\\\hline
%                       \end{tabular}\smallskip\par
%                       |columnstrict=true|
%                     \end{minipage}
%                     \hskip15pt
%                     \begin{minipage}[b]{110pt}
%                       \centering
%                       \begin{tabular}{|c|c|c|}\hline
%                       1 & 3 & 5\\\hline
%                       2 & 4 & \\\hline
%                       &&\\\hline
%                       \end{tabular}\smallskip\par
%                       |columnstrict=false|
%                     \end{minipage}
%        \item[openright] This option puts an empty page before the
%                      first logical page. In combination with |nup=2x1|,
%                      |nup=2x2|, etc., this means that the first page
%                      is on the right side. The same effect can be
%                      achieved with the |pages| option, if an empty page
%                      is inserted in front of the first page.
%                      Either `true' or `false' (or no value, which
%                      is equivalent to `true'). \df{openright=false}
%        \item[pagecommand] Declares \LaTeX{} commands, which are executed
%                      on each sheet of paper.
%                      \df{pagecommand=\{\bslash thispagestyle\{empty\}\}}
%         \item[turn] By default pages in landscape format are displayed 
%                     in landscape orientation (if the PDF viewer supports
%                     this). With \texttt{turn=false} this can be prohibited.
%                     Either `true' or `false' (or no value, which
%                     is equivalent to `true'). \df{turn=true}
%         \item[noautoscale] \hskip0pt\marginpar{\marginstyle new}
%                     By default pages are scaled automatically.
%                     This can be suppressed with the 
%                     \texttt{noautoscale} option.
%                     In combination with the \texttt{scale} option
%                     (from graphicx) you has full control over
%                     the scaling process.
%                     Either `true' or `false' (or no value, which
%                     is equivalent to `true'). \df{noautoscale=false}
%         \item[fitpaper] \hskip0pt\marginpar{\marginstyle new}
%                      Adjusts the paper size to the one of the
%                      inserted document.
%                     Either `true' or `false' (or no value, which
%                     is equivalent to `true'). \df{fitpaper=false}
%      \end{description}
%
%    \item Hypertext options:
%
%      \begin{description}
%      \makeatletter\@beginparpenalty=10000\makeatother
%        \item[link] \hskip0pt\marginpar{\marginstyle changed}
%                    Inserted pages become a target
%                    of a hyperlink. The name of the link is 
%                    `\texttt{\meta{filename}.\meta{page number}}'.
%                    The file extension of \meta{filename} 
%                    \emph{must not} be stripped.
%                    Either `true' or `false' (or no value, which
%                    is equivalent to `true').
%                    \df{link=false}
%        \item[linkname] Changes the default linkname created by the option 
%                        \texttt{link}. Instead of \meta{filename} the
%                        value of this option is used. E.\,g. \texttt{
%                        linkname=mylink} produces the linknames
%                        `\texttt{mylink.\meta{page number}}'.\\
%                        \df{linkname=\meta{filename}.\meta{page number}}
%        \item[thread] Combines pages to an article thread.
%                     Either `true' or `false' (or no value, which
%                     is equivalent to `true'). \df{thread=false}
%        \item[threadname] Several threads are distinguished by their names.
%                          By default this name is equal to the filename. It
%                          can be changed with this option. This is useful if 
%                          the same file is inserted twice or more times and
%                          should not be combined to one single thread.
%                          \df{threadname=\meta{filename}}
%        \item[linktodoc] Lets the inserted pages be hyperlinks to the
%                         document from which they were extracted.
%                         Either `true' or `false' (or no value, which
%                         is equivalent to `true'). \df{linktodoc=false}
%      \end{description}
%
%    \item Additional hypertext options:
%
%      \begin{description}
%      \makeatletter\@beginparpenalty=10000\makeatother
%        \item[linkfit] Specifies the way the viewer displays a linked page.
%                       This option changes the default behavior of the
%                       option \texttt{link}.
%                       Possible values are: \texttt{fitb}, \texttt{fith},
%                       \texttt{fitv}, \texttt{fitr}, 
%                       \texttt{xyz zoom \meta{integer}}, etc.
%                       These are destinations (|\pdfdest|) like they are
%                       described in \cite{pdftex}.
%                       \df{linkfit=fitr}
%        \item[linktodocfit] By default the option \texttt{linktodoc} opens
%                            the page in `Fit in Window' view.
%                            Another view can be specified with this option.
%                            Possible values are the legal PDF tokens:
%                            \texttt{/FitH} \meta{top}, \texttt{/FitV}
%                            \meta{left}, etc.
%                            (See \cite{PDFRef} for more details.)
%                            \df{linktodocfit=/Fit}
%        \item[linkfilename] Sets the name (with path) of the file to be
%                            linked to by the option \texttt{linktodoc}.
%                            You will hardly ever need this option.
%                            \df{linkfilename=\meta{filename}}
%      \end{description}
%
%    \item Obsolete options:
%
%      \begingroup
%      \renewcommand*\descriptionlabel[1]{\hspace\labelsep
%                                \normalfont\slshape #1}
%      \begin{description}{\small}
%      \makeatletter\@beginparpenalty=10000\makeatother
%        \item[deltax] Puts some horizontal space between the
%                     logical pages. The argument should be one dimension,
%                     separated by space. \df{deltax=0}.
%        \item[deltay] Puts some vertical space between the
%                     logical pages. The argument should be one dimension,
%                     separated by space. \df{deltay=0}.
%      \end{description}
%      \endgroup
%    \end{itemize}
% 
%
%    \begin{figure}
%      \centering\setlength{\unitlength}{.7mm}
\def\thick{.7pt}
\def\thin{.2pt}
%^^A\fbox{
\begin{picture}(145,238)
  \small
\put(0,33){
  \linethickness{\thick}
  \put(0,0){\framebox(145,205){}}
  \linethickness{\thin}
  \put(10,15){\dashbox(35,49){}}
  \put(55,15){\dashbox(35,49){}}
  \put(100,15){\dashbox(35,49){}}
  \put(10,78){\dashbox(35,49){}}
  \put(55,78){\dashbox(35,49){}}
  \put(100,78){\dashbox(35,49){}}
  \put(10,141){\dashbox(35,49){}}
  \put(55,141){\dashbox(35,49){}}
  \put(100,141){\dashbox(35,49){}}
  \put(5,7){
    \put(10,15){\framebox(35,49){}}
    \put(55,15){\framebox(35,49){}}
    \put(100,15){\framebox(35,49){}}
    \put(10,78){\framebox(35,49){}}
    \put(55,78){\framebox(35,49){}}
    \put(100,78){\framebox(35,49){}}
    \put(10,141){\framebox(35,49){}}
    \put(55,141){\framebox(35,49){}}
    \put(100,141){\framebox(35,49){}}
    \put(45,36){\vector(1,0){10}}
    \put(55,36){\vector(-1,0){10}}
    \put(44.5,38){\textit{delx}}
    \put(90,36){\vector(1,0){10}}
    \put(100,36){\vector(-1,0){10}}
    \put(89.5,38){\textit{delx}}
    \put(27,64){\vector(0,1){14}}
    \put(27,78){\vector(0,-1){14}}
    \put(15,70){\textit{dely}}
    \put(27,127){\vector(0,1){14}}
    \put(27,141){\vector(0,-1){14}}
    \put(15,133){\textit{dely}}
  }
  \put(10,15){\vector(1,0){5}}
  \put(15,15){\vector(0,1){7}}
  \put(8,8){\textit{offx}}
  \put(17,17){\textit{offy}}
}
%
%^^A legend
%
\setlength{\fboxsep}{4pt}
\put(72.5,10){
  \makebox(0,0){
    \fbox{
      \begin{tabular}{l@{\hskip20pt}l}
         \linethickness{\thick}^^A
         \raisebox{.8ex}{\line(1,0){5}} physical sheet of paper
       &
         |delta=|\meta{delx} \meta{dely}
       \\
         \linethickness{\thin}^^A
         \raisebox{.8ex}{\line(1,0){5}} logical pages
       &
         |offset=|\meta{offx} \meta{offy}
       \\
         \raisebox{.8ex}{\line(1,0){1}\hskip.55mm\line(1,0){1}
           \hskip.55mm\line(1,0){1}} centered position
       &
       \\
       \end{tabular}
  }}
}

\end{picture}
%^^A}
%      \caption{Layout}
%      \label{fig:layout}
%    \end{figure}
%
%
%    \noindent
%    Internally the command |\includepdf| 
%    makes use of the |\includegraphics| command from the |graphicx| 
%    (actually |graphics|) package. This way it is possible to use
%    all the options of |\includegraphics|, too.
%    Options which are not interpreted by |\includepdf| 
%    are passed directly to |\includegraphics|.
%
%    Especially the `trim' and `clip' options of |\includegraphics|
%    are quite useful, if only parts of a page should be inserted.
%    (Maybe to cut off the header and footer of the inserted pages.)
%    Just use the `trim' and `clip' options as if they were options
%    of |\inlcudepdf|. They will be passed to |\includegraphics|
%    internally.
%    
%
%
%    \bigskip\noindent
%    \DescribeMacro{\includepdfmerge}
%    Inserts pages of several external PDF documents.
%
%    \medskip\noindent
%    |\includepdfmerge[|\meta{key=val}|]{|\meta{file-page-list}|}|
%
%    \begin{quote}
%    \begin{tabular}{@{}l@{\,--~}>{\raggedright}p{.73\linewidth}}
%        \meta{key=val}   & A comma separated list of options
%                   using the \meta{key}=\meta{value} syntax.
%                   \tabularnewline
%        \meta{file-page-list} & \meta{filename} [\textbf{,}\meta{page spec}]
%                   [\textbf{,}\meta{file-page-list}]\\
%                   A comma separated list of filenames and 
%                   optional \meta{page spec} specifiers.
%                   A \meta{page spec} can be everything the
%                   option |pages| accepts.
%    \end{tabular}
%    \end{quote}
%    The |\includepdfmerge| command uses the same options as 
%    |\includepdf| with
%    one exception. The option |pages| has no meaning for 
%    |\includepdfmerge|. Instead the \meta{page spec} specifier
%    is used to specify which pages should be inserted.
%    The \meta{page spec} specifier accepts the same values as 
%    the |pages| option. If no \meta{page spec} specifier is given,
%    only the first page will be inserted.
%
%
%
%    \bigskip\noindent
%    \DescribeMacro{\includepdfset}
%    If you need the same options for |\includepdf| all the time, it is
%    possible to define global options with |\includepdfset|. The
%    argument of |\includepdfset| is a comma separated list of options,
%    using the \meta{key}=\meta{value} syntax.
%    These options are processed each time |\includepdf| is called.
%    Local options (passed as an optional argument directly to |\includepdf|)
%    are overwriting global options:
%    \begin{quote}
%       |\includepdfset{|\meta{global options}|}|\\
%       |\includepdf[|\meta{local options}|]{pdf-file}|
%    \end{quote}
%    Only options specific to this package can be made global by
%    |\includepdfset|. Options of the graphicx package are not
%    concerned.
%
%    \bigskip\noindent
%    \DescribeMacro{\threadinfodict}
%    A thread information dictionary contains information about the
%    thread, such as its title, author, and creation date.
%    Use the macro |\threadinfodict| to set these entries.
%    This macro has one argument taking the thread information
%    dictionary in low-level PDF commands. See \cite{PDFRef} for
%    more information. This macro can be redefined. E.\,g.:
%    
%    |\renewcommand*{\threadinfodict}|\\\hspace*{\parindent}
%    |  {/I << /Title (My first thread) /Author (That's me!) >>}|
%    
%    
%
%    \subsection{The Layout}\label{sec:layout}
%    
%    The default layout can be changed by the options |delta|
%    and |offset|. Figure~\ref{fig:layout} shows the
%    meaning of these options.
%    
%    The inserted logical pages are being centered on the
%    sheet of paper by default. To displace them use the |offset|
%     option, which argument should be two dimensions. E.\,g.\
%    |offset=10mm 14mm| means that the logical pages are displaced
%    by 10\,mm in horizontal direction and by 14\,mm in vertical
%    direction.
%    
%    By default logical pages are being arranged side by side.
%    To put some space between them, use the |delta| option,
%    whose argument should be two dimensions.
%    Figure~\ref{fig:layout} shows the meaning of |delta|.
%    
%    The layout options |delta| and |offset|
%    \textit{always} refer to a sheet of paper in portrait
%    orientation. No matter whether you have set the 
%    |landscape| option to |true|, or not.
%    
%    If you are confused about horizontal\,(|x|)
%    and vertical\,(|y|) directions, just set the option
%    |turn=false|. Now your PDF viewer shows the pages in
%    the \textit{same} orientation as in figure~\ref{fig:layout}.
%    And the options |delta| and |offset| have
%    the \textit{same} meaning as in figure~\ref{fig:layout}.
%    Regardless of any other options.
%
%
%
%    \section{Required Packages}
%
%    The \texttt{pdfpages} package requires the following packages:
%    \begin{description}
%      \item[eso-pic]
%            \texttt{CTAN:macros/latex/contrib/supported/ms/contrib/}
%            Download the whole \texttt{ms/} directory, because 
%            eso-pic.sty requires everyshi.sty from that directory.
%      \item[graphicx, ifthen, calc]
%            These packages belong to the standard \LaTeX\
%            distribution.
%    \end{description}
%
%    \noindent
%    Furthermore it requires a recent version of:
%    \begin{description}
%      \item[pdftex.def]
%            \texttt{ http://www.tug.org/applications/pdftex/pdftex.def}
%    \end{description}
%
%    \noindent
%    Since pdf\TeX, Version 3.14159-1.00a-pretest-20010806,
%    PDF import has improved a lot. This
%    results in much smaller file sizes, faster processing and the
%    intuitively correct treatment of landscape pages.
%    The latest version of pdf\TeX\ can be found at:
%    \texttt{ftp://ftp.muni.cz/pub/tex/local/cstug/thanh/pdftex}.
%
%
%
%    \section{Acknowledgment}
%    I would like to thank \textsc{Rolf Niepraschk} and
%    \textsc{Heiko Oberdiek} for their useful
%    hints and suggestions. As well as \textsc{Ross Moore},
%    who encouraged me to implement the hypertext features.
%
%
%    \StopEventually{
%    \begin{thebibliography}{1}
%    \bibitem{pdftex}
%      H\`an Th\^e\llap{\raise 0.5ex\hbox{\'{}}} Th\`anh,
%      Sebastian Rahtz, Hans Hagen,
%      \textit{The pdfTeX user manual},
%      http://www.tug.org/applications/pdftex/pdftex-s.pdf
%    \bibitem{PDFRef} \textit{PDF Reference}, Second Edition,
%      Adobe Systems Incorporated,\\
%      http://partners.adobe.com/asn/developer/acrosdk/DOCS/PDFRef.pdf
%    \end{thebibliography}
%    }
%
%
%    \iffalse
%<*package>
%    \fi
%    \section{Implementation}
%    \begin{macrocode}
%    \end{macrocode}
%    \subsection{Required Packages and Version Checking}
%    Check for using PDF\LaTeX and producing PDF output. 
%    -- Note: 
%    If \texttt{graphics.sty} is loaded before this package,
%    \texttt{pdfoutput} may be undefined, even if pdfLaTeX
%    is used, because some \texttt{graphic.cfg} undefine
%    \texttt{pdfoutput} (\verb|\let\pdfoutput\@undefined|),
%    if the document contains the line \verb|\pdfoutput=0|.
%    \begin{macrocode}
\begingroup
\@ifundefined{pdfoutput}
    {\PackageError{pdfpages}
             {You're either not using pdfLaTeX\MessageBreak
              or have turn off PDF output.\MessageBreak
              `pdfpages' only works with pdfLaTeX\MessageBreak
              and PDF output}
             {Use `pdflatex' instead of `latex'\MessageBreak
              and set output to PDF (use: \pdfoutput = 1)!}
     \endinput}
    {\ifnum\pdfoutput=\z@
       \PackageError{pdfpages}
             {You don't produce PDF output, which\MessageBreak
              is required for the `pdfpages' package}
             {Set output to PDF. (use: \pdfoutput = 1)}
       \expandafter\endinput
     \fi}
\endgroup
%    \end{macrocode}
%    Required Packages.
%    \begin{macrocode}
\RequirePackage{ifthen, graphicx, calc, eso-pic}
%    \end{macrocode}
%    Check for \texttt{pdftex.def} version: This is done with
%    the following macro, which is derived from
%    \verb|\@ifpackagelater|
%    from the standard \LaTeX{} kernel.
%    \begin{macrocode}
\def\AM@ifdriverlater{\@ifl@ter{def}}
%    \end{macrocode}
%    Check for \texttt{pdftex.def} version \texttt{2000/05/12 v0.02u} 
%    or higher:
%    \begin{list}{$\ast$}{\versionlist}
%      \makeatletter\@beginparpenalty=10000\makeatother
%      \item support for multipage PDF files, option `page' added (HO).
%    \end{list}
%    \begin{macrocode}
\AM@ifdriverlater{pdftex}{2000/05/12}{}{%
    \PackageError{pdfpages}{%
         You have an old `pdftex.def'.\MessageBreak%
         There is no multipage support\MessageBreak%
         in `pdftex.def' < 0.02u.\MessageBreak%
         Update your `pdftex.def'}
         {Update your `pdftex.def'!}}
%    \end{macrocode}
%    Check for \texttt{pdftex.def} version \texttt{2000/09/14 v0.03d}
%     or higher:
%    \begin{list}{$\ast$}{\versionlist}
%      \makeatletter\@beginparpenalty=10000\makeatother
%      \item Fixes for `viewport' and `trim' (HO).
%      \item Clip support added for viewport and trim (HO).
%    \end{list}
%    \begin{macrocode}
\AM@ifdriverlater{pdftex}{2000/09/14}{}{%
    \PackageWarning{pdfpages}{%
         You have an old `pdftex.def'.\MessageBreak%
         There might be problems with\MessageBreak%
         the `trim' and `clip' options.\MessageBreak%
         Update your `pdftex.def'!\MessageBreak}}
%    \end{macrocode}
%    Checking for new versions of \texttt{eso-pic.sty}.
%    \begin{macrocode}
\def\AM@ifdriverlater{\@ifl@ter{sty}}
\AM@ifdriverlater{eso-pic}{2001/07/20}
  {\def\AM@AddToShipoutPicture{\AddToShipoutPicture*}%
   \def\AM@ClearShipoutPicture{\relax}}
  {\def\AM@AddToShipoutPicture{\AddToShipoutPicture}%
   \def\AM@ClearShipoutPicture{\ClearShipoutPicture}}
%    \end{macrocode}
%    \subsection{Package Options}
%    \begin{macrocode}
\newcommand*{\AM@draft}{false}
\DeclareOption{draft}{\def\AM@draft{true}}  
\DeclareOption{final}{\def\AM@draft{false}} 
\ProcessOptions  
%    \end{macrocode}
%    \subsection{Booleans}
%    \begin{macrocode}
\newboolean{AM@openright}\setboolean{AM@openright}{false}
\newboolean{AM@frame}\setboolean{AM@frame}{false}
\newboolean{AM@landscape}\setboolean{AM@landscape}{false}
\newboolean{AM@turn}\setboolean{AM@turn}{true}
\newboolean{AM@thread}\setboolean{AM@thread}{false}
\newboolean{AM@link}\setboolean{AM@link}{false}
\newboolean{AM@linktodoc}\setboolean{AM@linktodoc}{false}
\newboolean{AM@column}\setboolean{AM@column}{false}
\newboolean{AM@columnstrict}\setboolean{AM@columnstrict}{false}
\newboolean{AM@noautoscale}\setboolean{AM@noautoscale}{false}
\newboolean{AM@fitpaper}{\setboolean{AM@fitpaper}{false}
\newif\ifAM@linkfilename\AM@linkfilenamefalse
%    \end{macrocode}
%    \subsection{Commands}
%    Declare macro names and define default values.
%    \begin{macrocode}
\newcommand*{\AM@pagestemp}{1}
\newcommand*{\AM@pagecommand}{\thispagestyle{empty}}
\newcommand*{\AM@lscape@rot}{0}
\newcommand*{\AM@threadname}{}\newcommand{\AM@linkname}{}
\newcommand*{\AM@linkname@option}{}
\newcommand*{\AM@linkfit}{fitr}\newcommand{\AM@linktodocfit}{/Fit}
\newcommand*{\threadinfodict}{}
%    \end{macrocode}
%    \macro{\AM@setlscape}
%    Check, if `pdflscape.sty' is loaded.
%    This package is necessary, if landscape pages are produced
%    \textit{and} `thumbpdf.sty' is used.
%    \begin{macrocode}
\newif\ifAM@thumbpdf
\newif\ifAM@pdflscape
\newcommand*{\AM@setlscape}{}
\AtBeginDocument{%
  \@ifpackageloaded{thumbpdf}{\AM@thumbpdftrue}{\AM@thumbpdffalse}%
  \@ifpackageloaded{pdflscape}{\AM@pdflscapetrue}{\AM@pdflscapefalse}%
  \ifAM@thumbpdf
    \ifAM@pdflscape
      \def\AM@setlscape{\PLS@Rotate{90}}%
    \else
      \def\AM@setlscape{\pdfpageattr{/Rotate 90}\AM@thumbpdfwarning}%
    \fi
  \else
    \def\AM@setlscape{\pdfpageattr{/Rotate 90}}%
  \fi}
\newif\ifAM@thumbpdfwarning
\newcommand*{\AM@thumbpdfwarning}{%
  \ifAM@thumbpdfwarning\else\global\AM@thumbpdfwarningtrue
  \PackageWarning{pdfpages}{%
    You are using `thumbpdf.sty' but not\MessageBreak
    `pdflscape.sty'. \MessageBreak
    Please insert `pdflscape.sty' in  your document\MessageBreak
    to get a pleasant PDF document with thumbs.\MessageBreak}\fi}
%    \end{macrocode}
%    \macro{\AM@checkinteger}
%    Check if |#1| is a positiv integer.
%    \begin{macrocode}
\newif\ifAM@integer
\newcommand{\AM@checkinteger}{}
\def\AM@checkinteger#1{%
  \ifcat _\ifnum9<1#1_\else A\fi
    \AM@integertrue
  \else
    \AM@integerfalse
  \fi
}
%    \end{macrocode}
%    \macro{\includepdfset}
%    This command sets global options for |\includepdf|
%    and |\includepdfmerge|.
%    \begin{macrocode}
\newcommand*{\includepdfset}[1]{\setkeys{Gin}{#1}}
%    \end{macrocode}
%    \macro{\includepdf}
%    This command inserts any pages of an external 
%    PDF document.
%    \begin{macrocode}
\newcommand*{\includepdf}[2][]{%
  \begingroup
  \IfFileExists{#2}{\edef\AM@currentdocname{#2}}
    {\IfFileExists{#2.pdf}{\edef\AM@currentdocname{#2.pdf}}
       {\PackageError{pdfpages}{Cannot find file #2}{}}}%
  \setkeys{Gin}{#1}%
  \expandafter\AM@readlist\expandafter{\AM@pagestemp}%
  \AM@output{#1}%
  \endgroup
  \AM@ClearShipoutPicture
}
%    \end{macrocode}
%    \macro{\includepdfmerge}
%    This command merges pages of several PDF documents.
%    \begin{macrocode}
\newcommand*{\includepdfmerge}[2][]{%
  \begingroup
  \def\AM@currentdocname{}%
  \AM@readlist{#2}%
  \AM@output{#1}%
  \endgroup
  \AM@ClearShipoutPicture
}
%    \end{macrocode}
%    \macro{\AM@readlist}
%    \macro{\AM@readlisti}
%    \macro{\AM@checkrange}
%    \macro{\AM@checklast}
%    \macro{\AM@setnewdocname}
%    \macro{\AM@addtolist}
%    These macros parse a comma separated list, which contains
%    page numbers, page ranges, and/or document names. The
%    results are stored in two macros: |\AM@pagelist| which
%    contains only page numbers (page ranges are split into
%    several page numbers) and |\AM@doclist| which contains
%    the document names. One name for each page number in
%    |\AM@pagelist|.
%    \begin{macrocode}
\newif\ifAM@defaultone
\newcommand*{\AM@readlist}[1]{%
  \AM@defaultonefalse
  \def\AM@pagelist{}%
  \def\AM@doclist{}%
  \AM@readlisti#1,\END,
  \ifAM@defaultone\AM@addtolist{1}\fi
  \let\saveAM@doclist\AM@doclist
  \edef\AM@doclist{\expandafter\@gobble\saveAM@doclist}%
  \let\saveAM@pagelist\AM@pagelist
  \edef\AM@pagelist{\expandafter\@gobble\saveAM@pagelist}%
}
%    \end{macrocode}
%    \begin{macrocode}
\newcommand{\AM@readlisti}{}
\def\AM@readlisti#1,{%
  \def\AM@temp{#1}%
  \ifx\END#1
  \else
    \AM@checkrange#1-\END
    \expandafter\AM@readlisti
  \fi
}
%    \end{macrocode}
%    \begin{macrocode}
\newcommand{\AM@checkrange}{}
\def\AM@checkrange#1-#2\END{%
  \edef\AM@tempi{\AM@trim@space{#1}}%
  \ifx\\#2\\%
    \AM@checkinteger{\AM@tempi}%
    \ifAM@integer\AM@addtolist{\AM@tempi}%
    \else
      \expandafter\ifx\expandafter\\\AM@tempi\\\AM@addtolist{}%
      \else\AM@setnewdocname{\AM@tempi}\fi
    \fi
  \else
    \expandafter\ifx\expandafter\\\AM@tempi\\%
      \@tempcnta=1\relax
      \AM@checklast#2\END
    \else
      \AM@checkinteger{\AM@tempi}%
      \ifAM@integer
        \@tempcnta=\AM@tempi\relax
        \AM@checklast#2\END
      \else
        \AM@setnewdocname{\AM@temp}%
      \fi
    \fi
  \fi
}
%    \end{macrocode}
%    \begin{macrocode}
\newcommand{\AM@checklast}{}
\def\AM@checklast#1-\END{%
  \edef\AM@tempii{\AM@trim@space{#1}}%
  \expandafter\ifx\expandafter\\\AM@tempii\\
    \pdfximage{\AM@currentdocname}%
    \@tempcntb=\the\pdflastximagepages\relax
    \advance\@tempcntb\@ne
    \whiledo{\@tempcnta<\@tempcntb}{%
      \AM@addtolist{\the\@tempcnta}%
      \advance\@tempcnta\@ne
    }%
  \else
    \AM@checkinteger{\AM@tempii}%
    \ifAM@integer
      \@tempcntb=\AM@tempii\relax
      \advance\@tempcntb\@ne
      \whiledo{\@tempcnta<\@tempcntb}{%
        \AM@addtolist{\the\@tempcnta}%
        \advance\@tempcnta\@ne
      }%
    \else
      \AM@setnewdocname{\AM@temp}%
    \fi
  \fi
}
%    \end{macrocode}
%    \begin{macrocode}
\newcommand{\AM@setnewdocname}[1]{%
  \ifAM@defaultone
    \AM@addtolist{1}%
  \fi
  \edef\AM@temp{\AM@trim@space{#1}}%
  \IfFileExists{\AM@temp}{\edef\AM@currentdocname{\AM@temp}}
    {\IfFileExists{\AM@temp.pdf}{\edef\AM@currentdocname{\AM@temp.pdf}}
       {\PackageError{pdfpages}{Cannot find file \AM@temp}{}}}%
  \AM@defaultonetrue
}
%    \end{macrocode}
%    \begin{macrocode}
\newcommand{\AM@addtolist}[1]{%
  \let\saveAM@doclist\AM@doclist
  \edef\AM@doclist{\saveAM@doclist,\AM@currentdocname}%
  \let\saveAM@pagelist\AM@pagelist
  \edef\AM@pagelist{\saveAM@pagelist,#1}%
  \AM@defaultonefalse
}
%    \end{macrocode}
%    \macro{\AM@getfirst}
%    \macro{\AM@getfirsti}
%    \macro{\AM@getfirstii}
%    Parameter |#1| must be a macro containing a  csv-list. 
%    |\AM@getfirst| removes the first item of this list 
%    \textit{and} stores it in  |\toks@|.
%    \begin{macrocode}
\newboolean{AM@endoflist}\setboolean{AM@endoflist}{false}
\newcommand{\AM@getfirst}{}
\def\AM@getfirst#1{\expandafter\AM@getfirsti#1,\END#1\END}
%    \end{macrocode}
%    \begin{macrocode}
\newcommand{\AM@getfirsti}{}
\def\AM@getfirsti#1,#2\END#3\END{%
  \edef#3{#2}%
  \ifx\\#2\\%
    \setboolean{AM@endoflist}{true}%
    \def#3{}%
  \else
    \edef#3{\AM@getfirstii#2\END}%
  \fi
  \toks@{#1}%
}
%    \end{macrocode}
%    \begin{macrocode}
\newcommand{\AM@getfirstii}{}
\def\AM@getfirstii#1,\END{#1}
%    \end{macrocode}
%    \macro{\AM@readfirst}
%    Read first item of a list and store it in |\toks@|.
%    (Do not remove it.)
%    \begin{macrocode}
\newcommand{\AM@readfirst}{}
\newcommand{\AM@readfirsti}{}
\def\AM@readfirst#1{\expandafter\AM@readfirsti#1,\END}
\def\AM@readfirsti#1,#2\END{\toks@{#1}}
%    \end{macrocode}
%    \macro{\AM@getlast}
%    \macro{\AM@getlasti}
%    \macro{\AM@getlastn}
%    These macros are similar to |\AM@getfirst|. |\AM@getlast|
%    stores the last item in |\AM@lastitem| and removes it
%    from the list. |\AM@getlastn| does the same with the
%    last |n| items which are stored in |\AM@lastnitems|.
%    \begin{macrocode}
\newcommand{\AM@getlast}{}
\def\AM@getlast#1{%
  \let\AM@temp\relax
  \expandafter\AM@getlasti#1,\END
  \@ifundefined{AM@temp}{\def#1{}}{\edef#1{\AM@temp}}%
}
%    \end{macrocode}
%    \begin{macrocode}
\newcommand{\AM@getlasti}{}
\def\AM@getlasti#1,#2\END{%
  \ifx\\#2\\%
    \def\AM@lastitem{#1}%
  \else
    \@ifundefined{AM@temp}
      {\edef\AM@temp{#1}}
      {\edef\AM@temp{\AM@temp,#1}}%
    \AM@getlasti#2\END
  \fi
}
%    \end{macrocode}
%    \begin{macrocode}
\newcommand{\AM@getlastn}{}
\def\AM@getlastn#1#2{%
  \let\AM@templist\relax
  \@tempcnta=\z@
  \loop
  \ifnum\@tempcnta<#2
    \AM@getlast{#1}%
    \@ifundefined{AM@templist}
      {\edef\AM@templist{\AM@lastitem}}
      {\edef\AM@templist{\AM@lastitem,\AM@templist}}%
    \advance\@tempcnta\@ne
  \repeat
  \edef\AM@lastnitems{\AM@templist}%
}
%    \end{macrocode}
%    \macro{\AM@count}
%    \macro{\AM@counti}
%    Parameter |#1| must be a macro containing a csv-list.
%    |\AM@count| counts the items of this list, but it does
%    not count the last \textit{empty} items of this list.
%    The number of items is stored in |\@tempcnta|.
%    \begin{macrocode}
\newif\ifAM@again\AM@againtrue
\newcommand{\AM@count}{}
\def\AM@count#1{%
  \edef\AM@templist{#1}%
  \loop
    \AM@getlast{\AM@templist}%
    \expandafter\ifx\expandafter\\\AM@lastitem\\%
    \else
      \AM@againfalse
      \expandafter\ifx\expandafter\\\AM@templist\\%
        \edef\AM@templist{\AM@lastitem}%
      \else
        \edef\AM@templist{\AM@templist,\AM@lastitem}%
      \fi
      \@tempcnta=\z@
      \expandafter\AM@counti\AM@templist,\END,
    \fi
  \ifAM@again
  \repeat
}
%    \end{macrocode}
%    \begin{macrocode}
\newcommand{\AM@counti}{}
\def\AM@counti#1,{%
  \ifx\END#1
  \else
    \advance\@tempcnta\@ne
    \expandafter\AM@counti
  \fi
}
%    \end{macrocode}
%    \macro{\AM@resort}
%    To allow a more pleasant output in |column| mode, the
%    logical pages get resorted if they do not fill the
%    whole sheet of paper.
%    |\AM@lastnpages| must contain the pages and
%    |\AM@lastndocs| the corresponding documents which
%    should get resorted.
%
%    E.\,g.: After resorting a page that contains only 5 of 9
%    logical pages, it looks like (b) instead of (a).
%
%    \smallskip\par
%    \begin{minipage}{110pt}
%      \centering
%      \begin{tabular}{|c|c|c|}\hline
%      1 & 4 & \phantom{6}\\\hline
%      2 & 5 &\\\hline
%      3 &&\\\hline
%      \end{tabular}\smallskip\par
%      (a)~|columnstrict=false|
%    \end{minipage}
%    \hskip15pt
%    \begin{minipage}{110pt}
%      \centering
%      \begin{tabular}{|c|c|c|}\hline
%      1 & 3 & 5\\\hline
%      2 & 4 & \\\hline
%      &&\\\hline
%      \end{tabular}\smallskip\par
%      (b)~|columnstrict=true|
%    \end{minipage}
%    \smallskip
%    \begin{macrocode}
\newboolean{AM@resort@end}\setboolean{AM@resort@end}{false}
\newcommand{\AM@resort}{}
\def\AM@resort{%
  \def\AM@resortedpagelist{}%
  \def\AM@resorteddoclist{}%
  \def\AM@xnupi{\@ne}%
  \def\AM@ynupi{\@ne}%
%    \end{macrocode}
%    For the following algorithm a `last page' is
%    defined, which has the position (|xnuplast| , |ynuplast|).
%    It is marked with $\times$ in the figure below.
%    \smallskip\par
%    \begin{minipage}{110pt}
%      \centering
%      \begin{tabular}{|c|c|c|}\hline
%      1 & 3 & 5\\\hline
%      2 & 4 & $\times$\\\hline
%      &&\\\hline
%      \end{tabular}\smallskip\par
%    \end{minipage}
%    \begin{macrocode}
  \AM@count{\AM@lastnpages}%
  \def\AM@xnuplast{%
    \the\@tempcnta - (\the\@tempcnta / \AM@xnup * \AM@xnup) + 1}%
  \def\AM@ynuplast{1 + (\the\@tempcnta / \AM@xnup)}%
  \setlength{\@tempdima}{\p@*(\AM@xnuplast)}%
  \edef\AM@xnuplast{\strip@pt\@tempdima}%
  \setlength{\@tempdima}{\p@*(\AM@ynuplast)}%
  \edef\AM@ynuplast{\strip@pt\@tempdima}%
%    \end{macrocode}
%    \begin{macrocode}
  \whiledo{\not\boolean{AM@resort@end}}{%
    \@tempcnta=\AM@ynupi\relax
    \@tempcntb=\AM@ynuplast\relax
    \ifnum\@tempcnta=\@tempcntb
        \@tempcnta=\AM@xnupi\relax
        \@tempcntb=\AM@xnuplast\relax
        \ifnum\@tempcnta<\@tempcntb
          \AM@getfirst{\AM@lastndocs}%
          \edef\AM@firstdocitem{\the\toks@}%
          \AM@getfirst{\AM@lastnpages}%
          \edef\AM@resortedpagelist{\AM@resortedpagelist,\the\toks@}%
          \edef\AM@resorteddoclist{%
            \AM@resorteddoclist,\AM@firstdocitem}%
        \else
          \edef\AM@resortedpagelist{\AM@resortedpagelist,}%
          \edef\AM@resorteddoclist{\AM@resorteddoclist,\AM@phantomdoc}%
        \fi
    \else
      \ifnum\@tempcnta<\@tempcntb
        \AM@getfirst{\AM@lastndocs}%
        \edef\AM@firstdocitem{\the\toks@}%
        \AM@getfirst{\AM@lastnpages}%
        \edef\AM@resortedpagelist{\AM@resortedpagelist,\the\toks@}%
        \edef\AM@resorteddoclist{\AM@resorteddoclist,\AM@firstdocitem}%
      \else
        \edef\AM@resortedpagelist{\AM@resortedpagelist,}%
        \edef\AM@resorteddoclist{\AM@resorteddoclist,\AM@phantomdoc}%
      \fi
    \fi
    \@tempcnta=\AM@ynupi\relax
    \advance\@tempcnta\@ne
    \edef\AM@ynupi{\the\@tempcnta}%
    \@tempcntb=\AM@ynup\relax
    \ifnum\@tempcnta>\@tempcntb
      \@tempcntb=\AM@xnupi\relax
      \advance\@tempcntb\@ne
      \edef\AM@xnupi{\the\@tempcntb}%
      \def\AM@ynupi{\@ne}%
      \@tempcnta=\AM@xnup\relax
      \ifnum\@tempcntb>\@tempcnta
        \setboolean{AM@resort@end}{true}%
      \fi
    \fi
  }%
  \edef\AM@resortedpagelist{\expandafter\@gobble\AM@resortedpagelist}%
  \edef\AM@resorteddoclist{\expandafter\@gobble\AM@resorteddoclist}%
}
%    \end{macrocode}
%    \macro{\AM@output}
%    This is the main output routine used by |\includepdf|
%    and |\includepdfmerge|.
%    \begin{macrocode}
\newcommand{\AM@output}[1]{%
%    \end{macrocode}
%    Some macros may already be initialized by |\includepdfset|.
%    So do not reset them here.
%    \begin{macrocode}
  \@ifundefined{AM@xnup}
    {\def\AM@xnup{\@ne}\def\AM@ynup{\@ne}%
     \def\AM@xnupminusi{\z@}\def\AM@ynupminusi{\z@}}{}%
  \@ifundefined{AM@deltax}
    {\def\AM@deltax{-\fboxrule}\def\AM@deltay{-\fboxrule}}{}%
  \@ifundefined{AM@xoffset}
    {\def\AM@xoffset{\z@}\def\AM@yoffset{\z@}}{}%
  \def\AM@xpos{\z@}\def\AM@ypos{\z@}%
  \let\AM@linkname@option\relax
%    \end{macrocode}
%    Read in optional parameters.
%    \begin{macrocode}
  \setkeys{Gin}{draft=\AM@draft, #1}%
%    \end{macrocode}
%    Some preparations before starting the output.
%    \begin{macrocode}
  \AM@setphantom
  \AM@filluppagelist
  \ifthenelse{\boolean{AM@landscape}}
    {\def\AM@lscape@rot{90}%
     \@ifundefined{AM@shortnupsyntax}{}
       {\let\AM@temp\AM@xnup
        \let\AM@xnup\AM@ynup
        \let\AM@ynup\AM@temp}%
    }{}
%    \end{macrocode}
%    \begin{macrocode}
  \ifthenelse{\boolean{AM@column}%
              \and\not\boolean{AM@columnstrict}}
         {\AM@getlastn{\AM@doclist}{\AM@nup}%
          \edef\AM@lastndocs{\AM@lastnitems}%
          \AM@getlastn{\AM@pagelist}{\AM@nup}%
          \edef\AM@lastnpages{\AM@lastnitems}%
          \AM@resort
          \AM@count{\AM@pagelist}%
          \expandafter\ifx\expandafter\\\AM@pagelist\\%
            \edef\AM@pagelist{\AM@resortedpagelist}%
            \edef\AM@doclist{\AM@resorteddoclist}%
          \else
            \edef\AM@pagelist{\AM@pagelist,\AM@resortedpagelist}%
            \edef\AM@doclist{\AM@doclist,\AM@resorteddoclist}%
          \fi}{}
%    \end{macrocode}
%    \begin{macrocode}
  \ifthenelse{\boolean{AM@frame}}
         {\let\AM@fbox\fbox}
         {\let\AM@fbox\relax
          \setlength{\fboxrule}{\z@}}%
%    \end{macrocode}
%    \begin{macrocode}
  \ifthenelse{\boolean{AM@openright}}
         {\AM@openright}{}%
%    \end{macrocode}
%    \begin{macrocode}
  \ifthenelse{\boolean{AM@column}}
         {\def\AM@xnupi{\@ne}
          \def\AM@ynupi{\z@}}
         {\def\AM@xnupi{\z@}
          \def\AM@ynupi{\@ne}}%
%    \end{macrocode}
%    \begin{macrocode}
  \setlength{\fboxsep}{\z@}%
%    \end{macrocode}
%    Determine scaling factor:
%    The logical pages are scaled such, that they match on
%    one sheet of paper. To determine the scaling
%    factor all |scale| options given by the user must be 
%    discarded first (|\AM@remove@scale|).
%    But they are \textit{not} discarded when finally being 
%    shipped out.
%    \begin{macrocode}
    \AM@readfirst{\AM@pagelist}
    \edef\AM@page{\the\toks@}
    \AM@readfirst{\AM@doclist}
    \edef\AM@docname{\the\toks@}
    \ifAM@noautoscale
      \def\AM@scale{1}
    \else
      \AM@remove@scale{#1}
      \AM@isphantom
      \toks@\expandafter{\AM@options}
      \setbox\@tempboxa=\hbox{%
        \expandafter\includegraphics\expandafter[\the\toks@,
           page=\AM@page, angle=\AM@lscape@rot]{\AM@docname}}%
%    \end{macrocode}
%    Adjust the paper size.
%    \begin{macrocode}
    \newpage
    \ifthenelse{\boolean{AM@fitpaper}}
      {\setlength{\paperwidth}{\the\wd\@tempboxa}%
       \setlength{\paperheight}{\the\ht\@tempboxa}%
       \setlength{\pdfpagewidth}{\the\wd\@tempboxa}%
       \setlength{\pdfpageheight}{\the\ht\@tempboxa}%
       \@ifundefined{stockwidth}{}{%
         \setlength{\stockwidth}{\paperwidth}%
         \setlength{\stockheight}{\paperheight}%
         \setlength{\trimtop}{0pt}%
         \setlength{\trimedge}{0pt}}%
      }{}%
%    \end{macrocode}
%    Calculate required scaling in horizontal direction.
%    \begin{macrocode}
      \def\AM@xscale{%
        (\paperwidth - \tw@\fboxrule*\AM@xnup 
         - \AM@deltax * \AM@xnupminusi) /
        (\AM@xnup * \real{\strip@pt\wd\@tempboxa})
      }%
      \setlength{\@tempdima}{\AM@xscale}%
      \edef\AM@xscale{\strip@pt\@tempdima}%
%    \end{macrocode}
%    Calculate required scaling in vertical direction.
%    \begin{macrocode}
      \setlength{\@tempdimb}{\ht\@tempboxa + \dp\@tempboxa}%
      \def\AM@yscale{
        (\paperheight - \tw@\fboxrule*\AM@ynup 
         - \AM@deltay *\AM@ynupminusi) /
        (\AM@ynup * \real{\strip@pt\@tempdimb})
      }%
      \setlength{\@tempdima}{\AM@yscale}%
      \edef\AM@yscale{\strip@pt\@tempdima}%
%    \end{macrocode}
%    The lower value of |\AM@xscale| and |\AM@yscale| is used
%    as the overall scaling factor |\AM@scale|.
%    \begin{macrocode}
      \@tempdima=\AM@xscale\p@
      \@tempdimb=\AM@yscale\p@
      \ifdim\@tempdima<\@tempdimb
        \let\AM@scale\AM@xscale
      \else
        \let\AM@scale\AM@yscale
      \fi
    \fi
%    \end{macrocode}
%    Determine horizontal and vertical margins.
%    |\AM@xoffset| and |\AM@yoffset| are \textit{not} 
%    taken into account for this calculation.
%    \begin{macrocode}
    \let\AM@filelist@save\@filelist
    \setbox\@tempboxa=\hbox{%
      \includegraphics[page=\AM@page,
                       scale=\AM@scale, 
                       angle=\AM@lscape@rot, #1]{\AM@docname}}%
    \let\@filelist\AM@filelist@save
    \setlength{\@tempdimb}{\ht\@tempboxa + \dp\@tempboxa}%
    \def\AM@xmargin{%
      (\paperwidth - (\wd\@tempboxa+\tw@\fboxrule)*\AM@xnup
       - \AM@deltax * \AM@xnupminusi) * \real{.5}
    }%
    \def\AM@ymargin{%
      (\paperheight - (\@tempdimb+\tw@\fboxrule)*\AM@ynup
       - \AM@deltay * \AM@ynupminusi) * \real{.5}
    }%
    \setlength{\@tempdima}{\AM@xmargin}%
    \edef\AM@xmargin{\the\@tempdima}%
    \setlength{\@tempdima}{\AM@ymargin}%
    \edef\AM@ymargin{\the\@tempdima}%
%    \end{macrocode}
%    \begin{macrocode}
    \ifthenelse{\boolean{AM@landscape}}
         {\edef\AM@temp{\AM@xnup}
          \edef\AM@xnup{\AM@ynup}
          \edef\AM@ynup{\AM@temp}}
         {}%
%    \end{macrocode}
%    This loop arranges one or more PDF pages onto each sheet
%    of paper.
%    \begin{macrocode}
  \setboolean{AM@endoflist}{false}%
  \whiledo{\not\boolean{AM@endoflist}}{%
    \AM@getfirst{\AM@pagelist}%
      \edef\AM@page{\the\toks@}%
      \AM@isphantom
      \@tempcnta=\AM@page\relax
      \advance\@tempcnta\m@ne
      \edef\AM@page@linktodoc{\the\@tempcnta}%
    \AM@getfirst{\AM@doclist}%
      \edef\AM@docname{\the\toks@}%
      \ifAM@linkfilename
      \else
        \let\AM@linkfilename\AM@docname
      \fi
      \@ifundefined{AM@linkname@option}
        {\edef\AM@linkname{\the\toks@}}
        {\edef\AM@linkname{\AM@linkname@option}}%
    \newpage
%    \end{macrocode}
%    Each logical page has an element number of
%    |(\AM@xnupi,\AM@ynupi)|, where |(1,1)| indicates the upper
%    left position.
%    \begin{macrocode}
    \ifthenelse{\boolean{AM@column}}
      {\@tempcnta=\AM@ynupi\relax
       \@tempcntb=\AM@ynup\relax
       \advance\@tempcnta\@ne
       \ifnum\@tempcnta>\@tempcntb
         \@tempcnta=\@ne
         \@tempcntb=\AM@xnupi\relax
         \advance\@tempcntb\@ne
         \edef\AM@xnupi{\the\@tempcntb}%
       \fi
       \edef\AM@ynupi{\the\@tempcnta}%
       \@tempcnta=\AM@xnupi\relax
       \@tempcntb=\AM@xnup\relax
       \ifnum\@tempcnta>\@tempcntb
         \def\AM@xnupi{\@ne}%
       \fi
      }
      {\@tempcnta=\AM@xnupi\relax
       \@tempcntb=\AM@xnup\relax
       \advance\@tempcnta\@ne
       \ifnum\@tempcnta>\@tempcntb
         \@tempcnta=\@ne
         \@tempcntb=\AM@ynupi\relax
         \advance\@tempcntb\@ne
         \edef\AM@ynupi{\the\@tempcntb}%
       \fi
       \edef\AM@xnupi{\the\@tempcnta}%

       \@tempcnta=\AM@ynupi\relax
       \@tempcntb=\AM@ynup\relax
       \ifnum\@tempcnta>\@tempcntb
         \def\AM@ynupi{\@ne}%
       \fi
      }%
%    \end{macrocode}
%    Determine the position (coordinates) of the logical pages.
%    These are |\AM@xpos| and |\AM@ypos|.
%    \begin{macrocode}
    \def\AM@add{+}%
    \if@twoside\ifodd\c@page\else\def\AM@add{-}\fi\fi
    \setlength{\@tempdimb}{\ht\@tempboxa + \dp\@tempboxa}%
    \ifthenelse{\boolean{AM@landscape}}
      {\def\AM@xpos{%
         \AM@xmargin \AM@add\AM@xoffset + 
         (\wd\@tempboxa + \AM@deltax + \tw@\fboxrule)*(\AM@ynupi - 1)
       }%
       \def\AM@ypos{%
         \AM@ymargin + \AM@yoffset +
         (\@tempdimb + \AM@deltay + \tw@\fboxrule)*(\AM@xnupi - 1)
       }%
      }
      {\def\AM@xpos{%
         \AM@xmargin \AM@add\AM@xoffset + 
         (\wd\@tempboxa + \AM@deltax + \tw@\fboxrule)*(\AM@xnupi - 1)
       }%
       \def\AM@ypos{%
         \AM@ymargin + \AM@yoffset +
         (\@tempdimb + \AM@deltay + \tw@\fboxrule)*(\AM@ynup-\AM@ynupi)
       }%
      }%
    \setlength{\@tempdima}{\AM@xpos}%
    \edef\AM@xpos{\strip@pt\@tempdima}%
    \setlength{\@tempdima}{\AM@ypos}%
    \edef\AM@ypos{\strip@pt\@tempdima}%
    \setlength{\fboxsep}{\z@}%
%    \end{macrocode}
%    Ship out page.
%    \begin{macrocode}
    \ifAM@phantompage
      \def\AM@shippage{}%
    \else
      \toks@{#1}%
      \edef\AM@shippage{%
        \noexpand\put(\AM@xpos,\AM@ypos){%
        \noexpand\raisebox{\dp\@tempboxa}{%
        \AM@hyper\noexpand\AM@fbox{%
        \noexpand\includegraphics[%
                    page=\AM@page,scale=\AM@scale,
                    angle=\AM@lscape@rot, \the\toks@]{\AM@docname}}}}%
      }%
    \fi
    \AM@AddToShipoutPicture{\setlength{\unitlength}{\p@}%
      \global\let\AM@filelist@save\@filelist}%
    \expandafter\AM@AddToShipoutPicture\expandafter{%
      \AM@shippage
      \global\let\@filelist\AM@filelist@save
    }%
    \ifnum\AM@xnupi=\AM@xnup\relax
      \ifnum\AM@ynupi=\AM@ynup\relax
        \begingroup\AM@pagecommand\endgroup
        \hskip\z@
        \ifthenelse{\boolean{AM@turn}}
          {\ifthenelse{\boolean{AM@landscape}}
             {\AM@setlscape}{}}
          {}%
        \newpage
        \ifAM@pdflscape\PLS@Rotate{0}\fi
        \AM@ClearShipoutPicture
      \fi
    \fi
%    \end{macrocode}
%    \begin{macrocode}
  }% whiledo
}% AM@output
%    \end{macrocode}
%    \macro{\AM@remove@scale}
%    \macro{\AM@remove@scalei}
%    \macro{\AM@remove@scaleii}
%    Remove the \texttt{scale} item from the options list (|#1|)
%    and then save it in |\AM@options|.
%    \begin{macrocode}

\newif\ifAM@scale
\newcommand{\AM@remove@scale}{}
\def\AM@remove@scale#1{%
  \def\AM@options{}%
  \AM@remove@scalei#1,\END,
}
%    \end{macrocode}
%    \begin{macrocode}
\newcommand{\AM@remove@scalei}{}
\def\AM@remove@scalei#1,{%
  \ifx\END#1
  \else
    \AM@scalefalse
    \AM@remove@scaleii#1=\END
    \ifAM@scale
    \else
      \ifx\empty
        \AM@options\edef\AM@options{#1}%
      \else
        \toks@\expandafter{\AM@options,#1}%
        \edef\AM@options{\the\toks@}%
      \fi
    \fi
    \expandafter\AM@remove@scalei
  \fi
}
%    \end{macrocode}
%    \begin{macrocode}
\newcommand{\AM@remove@scaleii}{}
\def\AM@remove@scaleii#1=#2\END{%
  \ifx\\#2\\%
  \else
    \def\AM@temp{#1}%
    \edef\AM@temp{\AM@trim@space{\AM@temp}}%
    \ifthenelse{\equal{scale}{\AM@temp}}{\AM@scaletrue}{}%
  \fi
}
%    \end{macrocode}
%    \macro{\AM@trim@space}
%    \macro{\AM@trim@spacei}
%    \macro{\AM@trim@spaceii}
%    Remove leading and trailing spaces and save the result
%    in |\AM@trimmed|.
%    \begin{macrocode}
\newcommand{\AM@trim@space}{}\newcommand{\AM@trim@spacei}{}
\newcommand{\AM@trim@spaceii}{}
\def\AM@trim@space#1{%
  \expandafter\AM@trim@spacei\expandafter{\ifdim\z@=0pt#1}\fi}
\def\AM@trim@spacei#1{%
\AM@trim@spaceii#1 \END}
\def\AM@trim@spaceii#1 #2\END{#1}
%    \end{macrocode}
%    \macro{\AM@isphantom}
%    Check, if current page is an empty (phantom) page.
%    \begin{macrocode}
\newif\ifAM@phantompage
\newcommand*{\AM@isphantom}{%
  \expandafter\ifx\expandafter\\\AM@page\\%
    \let\AM@page\AM@phantompage
    \let\AM@ph\phantom
    \AM@phantompagetrue
  \else
    \let\AM@ph\relax
    \AM@phantompagefalse
  \fi
}
%    \end{macrocode}
%    \macro{\AM@hyper}
%    Add hypertext operations (targets, links and threads) to 
%    the inserted pages.
%    \begin{macrocode}
\newcommand{\AM@hyper}{%
%    \end{macrocode}
%    Page is a target for a link.
%    \begin{macrocode}
  \ifAM@link
    \pdfdest name {\AM@linkname.\AM@page} \AM@linkfit%
  \fi
%    \end{macrocode}
%    Page is linked to the original document.
%    \begin{macrocode}
  \ifAM@linktodoc
    \pdfannot width \wd\@tempboxa
              height \ht\@tempboxa
              depth \dp\@tempboxa
              {/Subtype /Link /Border [0 0 0]
               /A << /S /GoToR
                     /F (\AM@linkfilename)
                     /D [\AM@page@linktodoc \AM@linktodocfit]
                     /NewWindow true >>}%
  \fi
%    \end{macrocode}
%    Page is a bead in a thread.
%    \begin{macrocode}
  \ifAM@thread
      \pdfthread width \wd\@tempboxa
                 height \ht\@tempboxa
                 depth \dp\@tempboxa
                 attr {\threadinfodict}%
                 name {\AM@threadname}%
  \fi
}
%    \end{macrocode}
%    \macro{\AM@openright}
%    Insert an empty item at the beginning of |\AM@pagelist|.
%    \begin{macrocode}
\newcommand{\AM@openright}{}
\def\AM@openright{%
  \edef\AM@pagelist{,\AM@pagelist}%
  \edef\AM@doclist{\AM@phantomdoc,\AM@doclist}%
}
%    \end{macrocode}
%    \macro{\AM@countpagelist}
%    \macro{\AM@countpagelisti}
%    Count items in |\AM@pagelist|. The number
%    of items is stored in |\@tempcnta|.
%    \begin{macrocode}
\newcommand{\AM@countpagelist}{}
\def\AM@countpagelist{%
  \@tempcnta=\z@
  \expandafter\AM@countpagelisti\AM@pagelist,\END,
}
%    \end{macrocode}
%    \begin{macrocode}
\newcommand{\AM@countpagelisti}{}
\def\AM@countpagelisti#1,{%
  \ifx\END#1
  \else
    \advance\@tempcnta\@ne
    \expandafter\AM@countpagelisti
  \fi
}
%    \end{macrocode}
%    \macro{\AM@filluppagelist}
%    The number of items in |\AM@pagelist| and 
%    |\AM@doclist| should be a multiple of
%    $|\AM@xnup|\nobreak\times\nobreak|\AM@ynup|$.
%    If not, fill up the lists with empty items.
%    \begin{macrocode}
\newcommand{\AM@filluppagelist}{}
\def\AM@filluppagelist{%
  \@tempcnta=\AM@xnup\relax
  \@tempcntb=\AM@ynup\relax
  \multiply\@tempcnta\@tempcntb
  \edef\AM@nup{\the\@tempcnta}%
  \AM@countpagelist
  \@tempcntb=\@tempcnta
  \divide\@tempcntb\AM@nup
  \multiply\@tempcntb\AM@nup
  \advance\@tempcnta-\@tempcntb
  \ifnum\@tempcnta=\z@
    \else
      \advance\@tempcnta-\AM@nup
      \multiply\@tempcnta\m@ne
  \fi
  \whiledo{\@tempcnta > \z@}{%
    \let\saveAM@pagelist\AM@pagelist
    \edef\AM@pagelist{\saveAM@pagelist,}%
    \advance\@tempcnta\m@ne
    \let\saveAM@doclist\AM@doclist
    \edef\AM@doclist{\saveAM@doclist,\AM@phantomdoc}%
  }%
}
%    \end{macrocode}
%    \macro{\AM@setphantom}
%    \macro{\AM@setphantompage}
%    \macro{\AM@setphantomdoc}
%    The box measures of the phantom page are used to create
%    empty pages. |\AM@phantomdoc| is the name and
%    |\AM@phantompage| is the page number of the phantom page.
%    \begin{macrocode}
\newcommand*{\AM@setphantom}{%
  \expandafter\AM@setphantompage\AM@pagelist,\END,
  \expandafter\AM@setphantomdoc\AM@doclist,\END
}
%    \end{macrocode}
%    \begin{macrocode}
\newcommand{\AM@setphantompage}{}
\def\AM@setphantompage#1,{%
  \def\AM@gobble##1\END,{}
  \ifx\\#1\\%
    \expandafter\AM@setphantompage
  \else 
    \def\AM@phantompage{#1}%
    \expandafter\AM@gobble
  \fi
}
%    \end{macrocode}
%    \begin{macrocode}
\newcommand{\AM@setphatomdoc}{}
\def\AM@setphantomdoc#1,#2\END{\edef\AM@phantomdoc{#1}}
%    \end{macrocode}
%    \subsection{Options of \texttt{\protect\bslash includepdf}}
%    \begin{macrocode}
\define@key{Gin}{pages}{\def\AM@pagestemp{#1}}
\define@key{Gin}{nup}{\AM@parse@nup#1x\END}
\define@key{Gin}{deltax}{\AM@parse@deltax#1 \\}
\define@key{Gin}{deltay}{\AM@parse@deltay#1 \\}
\define@key{Gin}{delta}{\AM@parse@delta#1 \\}
\define@key{Gin}{offset}{\AM@parse@offset#1 \\}
\define@key{Gin}{frame}[true]{\setboolean{AM@frame}{#1}}
\define@key{Gin}{openright}[true]{\setboolean{AM@openright}{#1}}
\define@key{Gin}{pagecommand}{\def\AM@pagecommand{#1}}
\define@key{Gin}{landscape}[true]{\setboolean{AM@landscape}{#1}}
\define@key{Gin}{turn}[true]{\setboolean{AM@turn}{#1}}
\define@key{Gin}{column}[true]{\setboolean{AM@column}{#1}}
\define@key{Gin}{columnstrict}[true]{\setboolean{AM@columnstrict}{#1}}
\define@key{Gin}{noautoscale}[true]{\setboolean{AM@noautoscale}{#1}}
\define@key{Gin}{thread}[true]{\setboolean{AM@thread}{#1}}
\define@key{Gin}{threadname}{\def\AM@threadname{#1}}
\define@key{Gin}{link}[true]{\setboolean{AM@link}{#1}}
\define@key{Gin}{linkfit}{\def\AM@linkfit{#1}}
\define@key{Gin}{linkname}{\def\AM@linkname@option{#1}}
\define@key{Gin}{linktodoc}[true]{\setboolean{AM@linktodoc}{#1}}
\define@key{Gin}{linktodocfit}{\def\AM@linktodocfit{#1}}
\define@key{Gin}{linkfilename}{\pdfstringdef\AM@linkfilename{#1}%
                               \AM@linkfilenametrue}
\define@key{Gin}{fitpaper}[true]{\setboolean{AM@fitpaper}{#1}}
%    \end{macrocode}
%    \macro{\AM@parse@nup}
%    Parsing \texttt{nup} item.
%    \begin{macrocode}
\newcommand{\AM@parse@nup}{}
\def\AM@parse@nup#1x#2\END{%
  \ifx\\#2\\%
    \def\AM@shortnupsyntax{}%
    \def\AM@xnup{\@ne}%
    \def\AM@ynup{\@ne}%
    \ifnum#1=\tw@
      \def\AM@xnup{\tw@}%
      \def\AM@ynup{\@ne}%
    \fi
    \ifnum#1=4
      \def\AM@xnup{\tw@}%
      \def\AM@ynup{\tw@}%
    \fi
  \else
    \def\AM@xnup{#1}%
    \AM@parse@nupi#2%
  \fi
  \@tempcnta=\AM@xnup\relax \advance\@tempcnta\m@ne
  \edef\AM@xnupminusi{\the\@tempcnta}%
  \@tempcnta=\AM@ynup\relax \advance\@tempcnta\m@ne
  \edef\AM@ynupminusi{\the\@tempcnta}%
}
\def\AM@parse@nupi#1x{\def\AM@ynup{#1}}
%    \end{macrocode}
%    \macro{\AM@parse@deltax}
%    \macro{\AM@parse@deltay}
%    \macro{\AM@parse@offset}
%    Parsing arguments. These lines are derived
%    from \texttt{graphicx.sty}.
%    \begin{macrocode}
\def\AM@parse@deltax#1 #2\\{\AMGin@defaultbp\AM@deltax{#1}}
%    \end{macrocode}
%    \begin{macrocode}
\def\AM@parse@deltay#1 #2\\{\AMGin@defaultbp\AM@deltay{#1}}
%    \end{macrocode}
%    \begin{macrocode}
\def\AM@parse@delta#1 #2 #3\\{%
  \AMGin@defaultbp\AM@deltax{#1}%
  \AMGin@defaultbp\AM@deltay{#2}}
%    \end{macrocode}
%    \begin{macrocode}
\def\AM@parse@offset#1 #2 #3\\{%
  \AMGin@defaultbp\AM@xoffset{#1}%
  \AMGin@defaultbp\AM@yoffset{#2}}
%    \end{macrocode}
%    \begin{macrocode}
\def\AMGin@defaultbp#1#2{%
  \afterassignment\AMGin@def@bp\dimen@#2bp\relax{#1}{#2}}
%    \end{macrocode}
%    \begin{macrocode}
\def\AMGin@def@bp#1\relax#2#3{%
   \if!#1!%
     \def#2{#3bp}%
    \else
      \dimen@.99626\dimen@
      \edef#2{\strip@pt\dimen@ bp}%
    \fi
}
%    \end{macrocode}
%
%
%    \section{History}
%
%    \subsection*{2001/01/17 v0.1b}
%    \begin{itemize}
%      \item The first released version.
%    \end{itemize}
%
%    \subsection*{2001/02/03 v0.1c}
%    \begin{itemize}
%      \item Pages are being centered automatically.
%    \end{itemize}
%
%    \subsection*{2001/03/16 v0.1f}
%    \begin{itemize}
%      \item Option |pages| added.
%    \end{itemize}
%
%    \subsection*{2001/04/07 v0.1h}
%    \begin{itemize}
%      \item Hypertext operations added. (Hyperlinks, threads)
%    \end{itemize}
%
%    \subsection*{2001/11/12 v0.2}
%    \begin{itemize}
%      \item Support of |m|$\times$|n| logical pages on each
%        sheet of paper.
%      \item Options |column| and |columnstrict| added.
%      \item New command |\inlcudepdfmerge|.
%    \end{itemize}
%
%    \subsection*{2001/12/26 v0.2c}
%    \begin{itemize}
%      \item Option |fitpaper| added.
%    \end{itemize}
%
%    \iffalse
%</package>
%    \fi
%
%    \Finale
%
%    \iffalse
%<*example1>
\documentclass[a4paper,12pt, draft]{article}
\usepackage[final]{pdfpages}
%% Uncomment the following lines, if you want to produce thumbnails.
%%\usepackage{pdflscape}
%%\usepackage{thumbpdf}
\usepackage{verbatim}

\newcounter{example}
\setcounter{example}{1}

\newenvironment{example}
  {\par\vskip\topsep%
   \noindent\textbf{Example \arabic{example}:}%
   \stepcounter{example}%
   \par\vskip\topsep%
   \minipage{.9\linewidth}%
   \verbatim}
  {\endverbatim%
   \endminipage\par\vskip\topsep}

\begin{document}

\title{Demonstration of the \texttt{pdfpages} Package}
\author{Andreas MATTHIAS}
\maketitle

This is a demonstration of the \texttt{pdfpages} package.
It is \textit{not} the documentation of the package.
To get the documentation run: `latex pdfpages.dtx'

\tableofcontents

\section{Inserting Pages:}
\begin{example}
\includepdf[pages=3-5]{dummy}
\end{example}
\includepdf[pages=3-5]{dummy}

\section{Using the \texttt{nup} Option}
Arranging several logical pages on one sheet of paper.

\begin{example}
\includepdf[nup=1x2, landscape, pages=4-7]{dummy}
\end{example}
\includepdf[nup=1x2, landscape, pages=4-7]{dummy}

Use the option \texttt{turn=false}, if the pages
should not be displayed in landscape orientation.
(If the last two pages were not displayed in
landscape orientation, you use a PDF viewer that
does not support this option.)

\begin{example}
\includepdf[nup=1x2, landscape, 
            pages=4-7, turn=false]{dummy}
\end{example}
\includepdf[nup=1x2, landscape, 
            pages=4-7, turn=false]{dummy}

\begin{example}
\includepdf[nup=1x2, pages=1-4]{dummy-l}
\end{example}
\includepdf[nup=1x2, pages=1-4]{dummy-l}

\noindent
With the option \texttt{pages} special pages and
ranges of pages can be selected.

\begin{example}
\includepdf[nup=3x2, pages={1,4-6,8,10}]{dummy}
\end{example}
\includepdf[nup=3x2, pages={1,4-6,8,10}]{dummy}

\noindent
A column-major layout can be achieved with the
option \texttt{column}.

\begin{example}
\includepdf[nup=2x2, pages=1-4, column]{dummy}
\end{example}
\includepdf[nup=2x2, pages=1-4, column]{dummy}


\noindent
Sometimes it might be useful to put an empty page
at the beginning. 
\begin{example}
\includepdf[nup=2x2, pages={,5-7}]{dummy}
\end{example}
\includepdf[nup=2x2, pages={,5-7}]{dummy}


\section{Changing Layout}

To put some space between the logical pages, use the option
\texttt{delta}. 

Any options of \verb|\includegraphics| are allowed in
\verb|\includepdf| as well. See the \texttt{scale}
option in the next example.
\begin{example}
\includepdf[nup=2x2, pages=3-6, scale=.8,
            delta=8mm 11mm, frame]{dummy}
\end{example}
\includepdf[nup=2x2, pages=3-6, scale=.8,
            delta=8mm 11mm, frame]{dummy}

\noindent
By default the output is centered as you could see in
the last examples. With the \texttt{offset} option it is
possible to displace the output.

\begin{example}
\includepdf[nup=2x2, pages=3-6, scale=.8,
            offset=5mm 7mm, frame]{dummy}
\end{example}
\includepdf[nup=2x2, pages=3-6, scale=.8,
            offset=5mm 7mm, frame]{dummy}


\noindent
To remove the header and the footer of the inserted
document use the \texttt{trim} and \texttt{clip}
options of the \texttt{graphicx} package.

\begin{example}
\includepdf[nup=2x2, pages=1-4, 
            trim=0 40mm 0 40mm,
            clip, pagecommand={}]{dummy}
\end{example}


\noindent The option
\verb|pagecommand={\thispagestyle{empty}}|
is set by default. If you don't want the page style
to be set to empty, you can remove this default 
setting by using the \texttt{pagecommand} option
as follows:
\verb|pagecommand={}|.

\includepdf[nup=2x2, pages=1-4, 
            trim=0 40mm 0 40mm,
            clip, pagecommand={}]{dummy}


\noindent
You like the crazy way? Then try this one \texttt{;-)}
\begin{example}
\includepdf[pages={3,4}, nup=1x2,
            landscape, scale=1.1,
            angle=30, delta=0 -85mm]{dummy}
\end{example}
\includepdf[pages={3,4}, nup=1x2,
            landscape, scale=1.1,
            angle=30, delta=0 -85mm]{dummy}


\end{document}
%</example1>
%
%
%
%<*example2>
\documentclass[a4paper,12pt]{article}
\usepackage[final]{pdfpages}
%% Uncomment the following lines, if you want to produce thumbnails.
%%\usepackage{pdflscape}
%%\usepackage{thumbpdf}
\usepackage{verbatim}
\usepackage[bookmarksopen]{hyperref}

\newcounter{example}
\setcounter{example}{1}

\newenvironment{example}
  {\par\vskip\topsep%
   \noindent\textbf{Example \arabic{example}:}%
   \stepcounter{example}%
   \par\vskip\topsep%
   \minipage{.9\linewidth}%
   \verbatim}
  {\endverbatim%
   \endminipage\vskip\topsep}

\newcommand{\meta}[1]{\ensuremath\langle\texttt{#1}\ensuremath\rangle}
\newcommand{\link}[2]{\meta{#1}.\meta{#2}}

\newcommand*{\bookmark}[3][0]{%
  \pdfoutline user {<< /S /GoTo /D (#3) >>} count #1 {#2}}

\begin{document}

\title{Demonstration of the Hypertext Operations\\ of the \texttt{pdfpages} Package}
\author{Andreas MATTHIAS}
\maketitle

This is a demonstration of the \texttt{pdfpages} package.
It is \textit{not} the documentation of the package.
To get the documentation run: `latex pdfpages.dtx'

\tableofcontents

\section{Hyperlinks}
\subsection{Links to the inserted Pages}

Hyperlinks are created by using the option \texttt{link}
of the \verb|\includepdf| command. Each inserted page
gets a linkname consisting of the filename and
the page number: \link{filename}{page number}

\begin{example}
\includepdf[pages=1-2, link]{dummy}
\end{example}

In this example the two pages have the linknames
`\texttt{dummy.pdf.1}' and `\texttt{dummy.pdf.2}'.
Setting links to these
pages, can be done easily with the \verb|\hyperlink|
macro from the \texttt{hyperref.sty} package:

\bigskip
\verb|\hyperlink{dummy.pdf.1}{Page 1}|\hskip10pt \hyperlink{dummy.pdf.1}{Page 1}\par
\verb|\hyperlink{dummy.pdf.2}{Page 2}|\hskip10pt \hyperlink{dummy.pdf.2}{Page 2}
\bigskip

This way you can refer to the \hyperlink{dummy.pdf.1}{first}
and the \hyperlink{dummy.pdf.2}{second page}.

\includepdf[pages=1-2, link]{dummy}
\bookmark[4]{List of Hyperfigures}{}
\bookmark[0]{Page 1}{dummy.pdf.1}
\bookmark[0]{Page 2}{dummy.pdf.2}


Inserting the same page twice would result in two 
identical linknames. To prevent this use the option
\texttt{linkname} to specify another name for the links.

\begin{example}
\includepdf[pages=1-2, nup=1x2, landscape,
            link, linkname=mylink]{dummy}
\end{example}

\noindent
Now the links are called `\texttt{mylink.1}' and
`\texttt{mylink.2}'.

\bigskip
\verb|\hyperlink{mylink.1}{Page 1}|\hskip10pt \hyperlink{mylink.1}{Page 1}\par
\verb|\hyperlink{mylink.2}{Page 2}|\hskip10pt \hyperlink{mylink.2}{Page 2}

\includepdf[pages=1-2, nup=1x2, landscape,
            link, linkname=mylink]{dummy}
\bookmark[0]{Again Page 1}{mylink.1}
\bookmark[0]{Again Page 2}{mylink.2}


\subsection{Links to the original Document}

Each page can be a hyperlink to the document from
which it was extracted. This can be done with the
option \texttt{linktodoc}.

Click on the inserted pages and see what happens.

\begin{example}
\includepdf[pages=1-4, nup=2x2, linktodoc]{dummy}
\end{example}
\includepdf[pages=1-4, nup=2x2, linktodoc]{dummy}


\section{Article Threads}

In a PDF document one or more article threads may be defined.
An article thread is a logical connected sequence of content
items.

With the option \texttt{thread} the inserted pages become
an article thread.

\begin{example}
\includepdf[pages=1-4, nup=2x2, landscape, thread]{dummy-l}
\end{example}

In Acrobat Reader the mouse pointer changes to a hand with
a little arrow in it when moved over an article thread.
By clicking on the page you can easily follow the logical
structure of the article thread.

\includepdf[pages=1-4, nup=2x2, landscape, thread]{dummy-l}

\end{document}
%</example2>
%\fi
\endinput
