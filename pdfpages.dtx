%
%    \iffalse
%
% pdfpages.dtx Copyright (C) 2001 Andreas MATTHIAS
%
% This program may be distributed and/or modified under the
% conditions of the LaTeX Project Public License, either version 1.2
% of this license or (at your option) any later version.
% The latest version of this license is in
%   http://www.latex-project.org/lppl.txt
% and version 1.2 or later is part of all distributions of LaTeX 
% version 1999/12/01 or later.
%
% Please send error reports and suggestions for improvements to
%   Andreas MATTHIAS <amat@kabsi.at>.
%
%<*!package>
% ===================================================================
%  @LaTeX-package-file{
%    author         = {Andreas MATTHIAS},
%    version        = "0.1a",
%    date           = "12 January 2001",
%    filename       = "pdfpages.sty",
%    address        = "",
%    telephone      = "",
%    email          = "amat@kabsi.at",
%    codetable      = "ISO/ASCII",
%    keywords       = "pdfLaTeX, PDF, include, page",
%    dependences    = "ifthen, graphicx, eso-pic",
%    supported      = "yes",
%    docstring      = "Package for including pages of external
%                      PDF documents. (Requiring pdfLaTeX.)"
%  }
% ===================================================================
%</!package>
%
%<*dtx>
          \ProvidesFile{pdfpages.dtx}
%</dtx>
%    \fi
%
%    \CheckSum{389}
%%
%% \CharacterTable
%%  {Upper-case    \A\B\C\D\E\F\G\H\I\J\K\L\M\N\O\P\Q\R\S\T\U\V\W\X\Y\Z
%%   Lower-case    \a\b\c\d\e\f\g\h\i\j\k\l\m\n\o\p\q\r\s\t\u\v\w\x\y\z
%%   Digits        \0\1\2\3\4\5\6\7\8\9
%%   Exclamation   \!     Double quote  \"     Hash (number) \#
%%   Dollar        \$     Percent       \%     Ampersand     \&
%%   Acute accent  \'     Left paren    \(     Right paren   \)
%%   Asterisk      \*     Plus          \+     Comma         \,
%%   Minus         \-     Point         \.     Solidus       \/
%%   Colon         \:     Semicolon     \;     Less than     \<
%%   Equals        \=     Greater than  \>     Question mark \?
%%   Commercial at \@     Left bracket  \[     Backslash     \\
%%   Right bracket \]     Circumflex    \^     Underscore    \_
%%   Grave accent  \`     Left brace    \{     Vertical bar  \|
%%   Right brace   \}     Tilde         \~}
%%
%    \iffalse
%<package>\NeedsTeXFormat{LaTeX2e}
%<package>\ProvidesPackage{pdfpages}[2001/01/17 v0.1b Include pages of external PDF documents (AM)]
% \fi
%        \ProvidesFile{pdfpages.dtx}[2001/01/17 v0.1b Include pages of external PDF documents (AM)]
% \iffalse
%
%<*driver>
\documentclass[10pt,draft]{ltxdoc}
\usepackage{array}
  \newcommand{\versionlist}{%
    \setlength{\topsep}{0pt}%
    \setlength{\itemsep}{0pt}%
    \setlength{\leftmargin}{20pt}%
    \setlength{\parsep}{0pt}}
\begin{document}
  \DocInput{pdfpages.dtx}
\end{document}
%</driver>
%    \fi
%
%  \GetFileInfo{pdfpages.dtx}
%
%
%    \title{The \textsf{pdfpages} Package\thanks{This file
%           has version number \fileversion, last revised \filedate.}}
%    \author{Andreas MATTHIAS\\ \texttt{amat@kabsi.at}} 
%
%
%    \date{\filedate}
%
%    \maketitle
%
%    \begin{abstract}
%       \noindent
%       This package makes it easy to include pages of external
%       PDF documents. It is based on pdfLaTeX and does 
%       \textit{not} work with \LaTeX.
%    \end{abstract}
%
%    \tableofcontents
%
%
%    \section{Introduction}
%    When creating PDF documents it is sometimes useful to include
%    pages from other, external PDF documents. This can be done
%    with the \texttt{\bslash includegraphics} command from the
%    \texttt{graphics} package. But a simple
%    |\includegraphics{doc.pdf}| normally produces
%    `|Overfull \hbox|' and `|Overfull \vbox|' warnings,
%    because the size of the included page does not match to the
%    print space. To circumvent this, one have to scale (reduce)
%    the included page. However, this is sometimes not desirable.
%    
%    The \texttt{pdfpages} package makes it easy to include one or
%    more pages of external PDF documents without worrying about
%    the print space. It even gives you the possibility
%    to arrange two or four logical pages on one physical sheet
%    of paper. (Like `psnup'.)
%    
%    \section{Usage}
%    \subsection{Package Options}
%    |\usepackage[|\meta{options}|]{pdfpages}|
%    
%    \begin{quote}
%    \begin{tabular}{@{}r@{~}r@{~}>{\raggedright}p{.7\linewidth}}
%        \meta{option} --  & |final|:& Include pages. This is the
%                                      default. \tabularnewline[.25\baselineskip]
%                          & |draft|:& Do not include pages, but
%                                      print a box and the filename
%                                      instead.
%    \end{tabular}
%    \end{quote}
%    
%    
%    \subsection{Commands}
%    
%    \DescribeMacro{\includepdf}
%    Pages of external PDF documents are included as follows:
%    
%    \medskip\noindent
%    |\includepdf[|\meta{key=val}|]{|\meta{pdf-file}|}|
%    
%    \begin{quote}
%    \begin{tabular}{@{}l@{\,--~}>{\raggedright}p{.8\linewidth}}
%        \meta{key=val}   & A comma separated list of options
%                   using the \meta{key}=\meta{value} syntax.
%                   \tabularnewline
%        \meta{pdf-file} & The filename of the PDF document.
%    \end{tabular}
%    \end{quote}
%    
%    \noindent
%    Possible options specific to |\includepdf| are:
%    
%    \newcommand{\df}[1]{(Default: |#1|)}
%    
%    \begin{description}
%      \item[firstpage] Page number of the first page to include.
%                       \df{firstpage=1}
%      \item[lastpage] Page number of the last page to include.\\
%                      \df{lastpage=firstpage}
%      \item[nup] Put multiple logical pages on one physical sheet of paper.
%                 Possible values are: 1, 2 and 4. \df{nup=1}
%      \item[openright] Put first page on the right side. (Just in combination
%                       with |nup=2| and |nup=4|.)
%                       Either `true' or `false' (or no value, which
%                       is equivalent to `true'). \df{openright=false}
%      \item[frame] Put a frame around each logical page. Either `true'
%                   or `false' (or no value, which is equivalent to `true').
%                   \df{frame=false}
%      \item[offset] Displace the origin of the included page relatively
%                    to the lower left corner of the sheet. 
%                    The argument should be
%                    two dimensions, separated by space. \df{offset=0 0}
%      \item[deltax] Put some space between the logical pages in
%                    horizontal direction. The argument should be one
%                    dimension. \df{deltax=0}
%      \item[deltay] Put some space between the logical pages in
%                    vertical direction. The argument should be one
%                    dimension. \df{deltay=0}
%      \item[pagecommand] \LaTeX{} commands, which are executed on each
%                         physical sheet of paper.
%                         \df{pagecommand=\{\bslash thispagestyle\{empty\}\}}
%    \end{description}
%    
%    The |\includepdf| command calls the |\includegraphics| command from the
%    |graphicx| (actually |graphics|) package, which makes it possible to
%    use its options too.
%    All options which are not interpreted by |\includepdf| are passed to
%    |\includegraphics|.
%    This way it is possible e.\,g.~to use the `trim' and `clip' options 
%    of \verb|\includegraphics| by passing them to \verb|\includepdf|, to
%    include only parts of a page. (Maybe to cut off the header and footer
%    of the included pages.)
%    
%    
%    \DescribeMacro{\includepdfset}
%    If one needs the same options for |\includepdf| all the time, it is
%    possible to define global options with |\includepdfset|. The
%    argument of |\includepdfset| is a comma separated list of options,
%    using the \meta{key}=\meta{value} syntax.
%    These options are processed each time |\includepdf| is called.
%    Local options (passed as an optional argument directly to |\includepdf|)
%    are overwriting global options:
%    
%    |\includepdfset{|\meta{global options}|}|
%    
%    |\includepdf[|\meta{local options}|]{pdf-file}|
%    
%    
%    \section{Acknowledgment}
%    I would like to thank \textsc{Rolf Niepraschk} and
%    \textsc{Heiko Oberdiek} for their useful
%    hints and suggestions.
%
%
%
%    \StopEventually{}
%
%    \iffalse
%<*package>
%    \fi
%    \section{Implementation}
%    \begin{macrocode}
%    \end{macrocode}
%    \subsection{Checking}
%    Check for using pdfLaTeX. 
%    -- If \texttt{graphics.sty} is loaded before this package,
%    \texttt{pdfoutput} may be undefined, even if pdfLaTeX
%    is used, because some \texttt{graphic.cfg} undefine
%    \texttt{pdfoutput} (\verb|\let\pdfoutput\@undefined|),
%    if the document contains the line \verb|\pdfoutput=0|.
%    \begin{macrocode}
\begingroup
\@ifundefined{pdfoutput}
    {\PackageError{pdfpages}
            {You're either not using pdfLaTeX\MessageBreak
             or have turn off PDF output.\MessageBreak
             `pdfpages' works only with pdfLaTeX\MessageBreak
             and PDF output}
            {Use `pdflatex' instead of `latex'\MessageBreak
             and set output to PDF (use: \pdfoutput = 1)!}
     \expandafter\endinput
    }{}
%    \end{macrocode}
%    Check for producing PDF output.
%    \begin{macrocode}
\ifnum\pdfoutput=0
    \PackageError{pdfpages}
            {You don't produce PDF output, which\MessageBreak
             is required for the `pdfpages' package}
            {Set output to PDF. (use: \pdfoutput = 1)}
    \expandafter\endinput
\fi
\endgroup
%    \end{macrocode}
%    Check for \texttt{pdftex.def} version: This is done with
%    the following macro, which is derived from
%    \verb|\@ifpackagelater|
%    from the standard \LaTeX{} kernel.
%    \begin{macrocode}
\def\AM@ifdriverlater{\@ifl@ter{def}}
%    \end{macrocode}
%    Check for \texttt{pdftex.def} version \texttt{2000/05/12 v0.02u} or higher:
%    \begin{list}{$\ast$}{\versionlist}
%    \item support for multipage PDF files, option `page' added (HO).
%    \end{list}
%    \begin{macrocode}
\AtBeginDocument{%
  \AM@ifdriverlater{pdftex}{2000/05/12}{}{%
      \PackageError{pdfpages}{%
           You have an old `pdftex.def'.\MessageBreak%
           There is no multipage support\MessageBreak%
           in `pdftex.def' < 0.02u.\MessageBreak%
           Update your `pdftex.def'}
           {Update your `pdftex.def'!}}
%    \end{macrocode}
%    Check for \texttt{pdftex.def} version \texttt{2000/09/14 v0.03d} or higher:
%    \begin{list}{$\ast$}{\versionlist}
%    \item Fixes for `viewport' and `trim' (HO).
%    \item Clip support added for viewport and trim (HO).
%    \end{list}
%    \begin{macrocode}
  \AM@ifdriverlater{pdftex}{2000/09/14}{}{%
      \PackageWarning{pdfpages}{%
           You have an old `pdftex.def'.\MessageBreak%
           There might be problems with\MessageBreak%
           the `trim' and `clip' options.\MessageBreak%
           Update your `pdftex.def'!\MessageBreak}}
}
%    \end{macrocode}
%    \subsection{Required Packages}
%    \begin{macrocode}
\RequirePackage{ifthen, graphicx, eso-pic}
%    \end{macrocode}
%    \subsection{Package Options}
%    \begin{macrocode}
\newcommand{\AM@draft}{false}
\DeclareOption{draft}{\def\AM@draft{true}}  
\DeclareOption{final}{\def\AM@draft{false}} 
\ProcessOptions  
%    \end{macrocode}
%    \subsection{Counters and Booleans}
%    \begin{macrocode}
\newcounter{AM@firstpage}\setcounter{AM@firstpage}{1}
\newcounter{saveAM@firstpage}
\newcounter{AM@lastpage}\setcounter{AM@lastpage}{2}
\newcounter{saveAM@lastpage}
\newcounter{AM@nup}\setcounter{AM@nup}{1}
\newcounter{saveAM@nup}
\newboolean{AM@openright}\setboolean{AM@openright}{false}
\newboolean{AM@frame}\setboolean{AM@frame}{false}
\newcounter{AM@page}\setcounter{AM@page}{1}
\renewcommand{\theAM@page}{\the\value{AM@page}}
%    \end{macrocode}
%    \subsection{Commands}
%    \begin{macrocode}
\newcommand{\AM@xoffset}{0}\newcommand{\saveAM@xoffset}{\AM@xoffset}
\newcommand{\AM@yoffset}{0}\newcommand{\saveAM@yoffset}{\AM@yoffset}
\newcommand{\AM@deltax}{0}\newcommand{\saveAM@deltax}{\AM@deltax}
\newcommand{\AM@deltay}{0}\newcommand{\saveAM@deltay}{\AM@deltay}
\newcommand{\AM@pagecommand}{\thispagestyle{empty}}
\newcommand{\saveAM@pagecommand}{\AM@pagecommand}
\newcommand{\AM@fbox}{}\newcommand{\saveAM@fbox}{\AM@fbox}
\newsavebox{\AM@pdfpagebox}
%    \end{macrocode}
%    \macro{\includepdfset}
%    This command sets global options for \verb|\includepdf|.
%    \begin{macrocode}
\newcommand*{\includepdfset}[1]{\setkeys{Gin}{#1}}
%    \end{macrocode}
%    \macro{\includepdf}
%    This command includes one or more pages of an external 
%    PDF document.
%    \begin{macrocode}
\newcommand*{\includepdf}[2][]{%
  \begingroup% setkey
%    \end{macrocode}
%    Save counters.
%    \begin{macrocode}
  \setcounter{saveAM@firstpage}{\value{AM@firstpage}}
  \setcounter{saveAM@lastpage}{\value{AM@lastpage}}
  \setcounter{saveAM@nup}{\value{AM@nup}}
%    \end{macrocode}
%    Read in optional parameters.
%    \begin{macrocode}
  \setkeys{Gin}{draft=\AM@draft, #1}
  \ifthenelse{\boolean{AM@frame}}
         {\let\AM@fbox\fbox}{\let\AM@fbox\relax}
%    \end{macrocode}
%    Arrange logical PDF pages on one physical sheet of paper.
%    \begin{macrocode}
  \setcounter{AM@page}{\value{AM@firstpage}}
  \whiledo{\value{AM@page} < \value{AM@lastpage}}{%
    \newpage
    \AddToShipoutPicture{%
      \setlength{\unitlength}{1bp}
      \setlength{\fboxsep}{0pt}
%    \end{macrocode}
%    One PDF page per sheet. (\texttt{nup=1})
%    \begin{macrocode}
      \ifthenelse{\value{AM@nup} = 1}{%
        \savebox{\AM@pdfpagebox}{%
                 \includegraphics[page=\theAM@page, #1]{#2}}
        \put(\AM@xoffset,\AM@yoffset){%
             \AM@fbox{\usebox{\AM@pdfpagebox}}}
      }{}
%    \end{macrocode}
%    Two PDF pages per sheet. (\texttt{nup=2})
%    \begin{macrocode}
      \ifthenelse{\value{AM@nup} = 2}{%
        \savebox{\AM@pdfpagebox}{\AM@fbox{%
                 \includegraphics[page=\theAM@page,
                                  angle=90, scale=.7071, #1]{#2}}}
        \let\AM@xposi\AM@xoffset
        \let\AM@yposi\AM@yoffset
        \let\AM@yposii\AM@yoffset
          \dimen@\AM@yposii\p@\advance\dimen@\ht\AM@pdfpagebox%
                             \edef\AM@yposii{\strip@pt\dimen@}%
          \dimen@\AM@yposii\p@\advance\dimen@\AM@deltay\p@%
                             \edef\AM@yposii{\strip@pt\dimen@}%
        \ifthenelse{\boolean{AM@openright}\and%
                    \value{AM@page}=\value{AM@firstpage}}{}{%
          \put(\AM@xposi,\AM@yposi){\usebox{\AM@pdfpagebox}}
          \stepcounter{AM@page}}
        \ifthenelse{\value{AM@page} < \value{AM@lastpage}}{%
          \put(\AM@xposi,\AM@yposii){\AM@fbox{%
               \includegraphics[page=\theAM@page,
                                angle=90, scale=.7071, #1]{#2}}}}{}
      }{}
%    \end{macrocode}
%    Four PDF pages per sheet. (\texttt{nup=4})
%    \begin{macrocode}
      \ifthenelse{\value{AM@nup} = 4}{%
        \savebox{\AM@pdfpagebox}{\AM@fbox{%
                 \includegraphics[page=\theAM@page,
                                  scale=.5, #1]{#2}}}
        \let\AM@xposi\AM@xoffset
        \let\AM@xposii\AM@xoffset
          \dimen@\AM@xposii\p@\advance\dimen@\wd\AM@pdfpagebox%
                             \edef\AM@xposii{\strip@pt\dimen@}%
          \dimen@\AM@xposii\p@\advance\dimen@\AM@deltax\p@%
                             \edef\AM@xposii{\strip@pt\dimen@}%
          \dimen@\AM@xposii\p@\advance\dimen@ -\fboxrule%
                             \edef\AM@xposii{\strip@pt\dimen@}%
        \let\AM@yposi\AM@yoffset
        \let\AM@yposii\AM@yoffset
          \dimen@\AM@yposii\p@\advance\dimen@\ht\AM@pdfpagebox%
                             \edef\AM@yposii{\strip@pt\dimen@}%
          \dimen@\AM@yposii\p@\advance\dimen@\AM@deltay\p@%
                             \edef\AM@yposii{\strip@pt\dimen@}%
        \ifthenelse{\boolean{AM@openright}\and%
                    \value{AM@page}=\value{AM@firstpage}}{}{%
          \put(\AM@xposi,\AM@yposii){\usebox{\AM@pdfpagebox}}
          \stepcounter{AM@page}}
        \ifthenelse{\value{AM@page} < \value{AM@lastpage}}{%
          \put(\AM@xposii,\AM@yposii){\AM@fbox{%
               \includegraphics[page=\theAM@page,
                                scale=.5, #1]{#2}}}
          \stepcounter{AM@page}}{}
        \ifthenelse{\value{AM@page} < \value{AM@lastpage}}{%
          \put(\AM@xposi,\AM@yposi){\AM@fbox{%
               \includegraphics[page=\theAM@page,
                                scale=.5, #1]{#2}}}
          \stepcounter{AM@page}}{}
        \ifthenelse{\value{AM@page} < \value{AM@lastpage}}{%
          \put(\AM@xposii,\AM@yposi){\AM@fbox{%
               \includegraphics[page=\theAM@page,
                                scale=.5, #1]{#2}}}}{}
      }{}
    }% AddToShipoutPicture
    \AM@pagecommand
    \hskip0pt
    \newpage
    \stepcounter{AM@page}
    \ClearShipoutPicture
  }% whiledo
%    \end{macrocode}
%    Write back saved counters.
%    \begin{macrocode}
  \setcounter{AM@firstpage}{\value{saveAM@firstpage}}
  \setcounter{AM@lastpage}{\value{saveAM@lastpage}}
  \setcounter{AM@nup}{\value{saveAM@nup}}
  \endgroup% setkey
  \ClearShipoutPicture
}% includepdf
%    \end{macrocode}
%    \subsection{Options of \texttt{\protect\bslash includepdf}}
%    \begin{macrocode}
\define@key{Gin}{firstpage}{%
                   \setcounter{AM@firstpage}{#1}}
\define@key{Gin}{lastpage}{%
                   \setcounter{AM@lastpage}{#1}%
                   \stepcounter{AM@lastpage}}
\define@key{Gin}{offset}{\AM@parse@offset#1 \\}
\define@key{Gin}{nup}{\setcounter{AM@nup}{#1}}
\define@key{Gin}{deltax}{\AM@parse@deltax#1 \\}
\define@key{Gin}{deltay}{\AM@parse@deltay#1 \\}
\define@key{Gin}{frame}[true]{\setboolean{AM@frame}{#1}}
\define@key{Gin}{openright}[true]{\setboolean{AM@openright}{#1}}
\define@key{Gin}{pagecommand}{\renewcommand{\AM@pagecommand}{#1}}
%    \end{macrocode}
%    Parsing option entries. These lines are derived
%    from \texttt{graphicx.sty}.
%    \begin{macrocode}
\def\AM@parse@deltax#1 #2\\{\AMGin@defaultbp\AM@deltax{#1}}
\def\AM@parse@deltay#1 #2\\{\AMGin@defaultbp\AM@deltay{#1}}
\def\AM@parse@offset#1 #2 #3\\{%
  \AMGin@defaultbp\AM@xoffset{#1}%
  \AMGin@defaultbp\AM@yoffset{#2}}
\def\AMGin@defaultbp#1#2{%
  \afterassignment\AMGin@def@bp\dimen@#2bp\relax{#1}{#2}}
\def\AMGin@def@bp#1\relax#2#3{%
   \if!#1!%
     \def#2{#3}%
    \else
      \dimen@.99626\dimen@
      \edef#2{\strip@pt\dimen@}%
    \fi}
%    \end{macrocode}
%    \iffalse
%</package>
%    \fi
%
%    \Finale
%
%    \iffalse
%<*example>
\documentclass[a4paper,12pt, draft]{article}
\usepackage[final]{pdfpages}
\usepackage{verbatim}

\newcounter{example}
\setcounter{example}{1}

\newenvironment{example}
  {\par\vskip\topsep%
   \noindent\textbf{Example \arabic{example}:}%
   \stepcounter{example}%
   \par\vskip\topsep%
   \minipage{.8\linewidth}%
   \verbatim}
  {\endverbatim%
   \endminipage}

\begin{document}

\title{Demonstration of the \texttt{pdfpages} Package}
\author{Andreas MATTHIAS}
\maketitle

This is a demonstration of the \texttt{pdfpages} package.
It is \textit{not} the documentation of the package.
To get the documentation run: `latex pdfpages.dtx'

\tableofcontents

\section{Including Pages:}
\begin{example}
\includepdf[firstpage=3, lastpage=5]{dummy}
\end{example}
\includepdf[firstpage=3, lastpage=5]{dummy}

\section{Using the \texttt{nup} Option}
Arranging two logical pages on one physical sheet of paper.

\begin{example}
\includepdf[nup=2,
            firstpage=4, lastpage=7]{dummy}
\end{example}
\includepdf[nup=2,
            firstpage=4, lastpage=7]{dummy}

\noindent
Arranging four logical pages on one physical sheet of paper.

\begin{example}
\includepdf[nup=4,
            firstpage=4, lastpage=7]{dummy}
\end{example}
\includepdf[nup=4,
            firstpage=4, lastpage=7]{dummy}

\noindent
Sometimes it might be useful to put the first page
on the right side. This can be done with the \texttt{openright} Option.
\begin{example}
\includepdf[nup=4,
            firstpage=5, lastpage=7,
            openright]{dummy}
\end{example}
\includepdf[nup=4,
            firstpage=5, lastpage=7,
            openright]{dummy}

\section{Changing Layout}

To put same space between the logical pages, use the options
\texttt{deltax} and \texttt{deltay}, which set the
horizontal and vertical space between the pages.
To displace the origin (the lower left corner) use the
\texttt{offset} option.

Any options of \verb|\includegraphics| are allowed in
\verb|\includepdf| as well. See the \texttt{scale}
option in the next example.

\begin{example}
\includepdf[nup=4, firstpage=3, lastpage=6,
            scale=.8, offset=11mm 20mm,
            deltax=20mm, deltay=20mm,
            frame]{dummy}
\end{example}

\small
A short explanation about \texttt{scale} in the above
example:
To put four pages on one physical sheet of paper, the
pages are scaled by $0.5$. This is done internally in
\verb|\includepdf|. The entry \texttt{scale=.8} in the
above example is an additional scale operation. This
means that on the whole the pages are scale by $0.5$
\textit{and} $0.8$. 

If the included paper format is A4 (\texttt{a4paper}),
which has a width of $210\,$mm, the pages are scaled to
a width of $84\,$mm:
$$210\,\mathrm{mm} \times 0.5 \times 0.8 = 84\,\mathrm{mm}$$
\normalsize

\includepdf[nup=4, firstpage=3, lastpage=6,
            scale=.8, offset=11mm 20mm,
            deltax=20mm, deltay=20mm,
            frame]{dummy}

\noindent
To remove the header and the footer of the included
document use the \texttt{trim} and \texttt{clip}
options.

\begin{example}
\includepdf[page=3,
            trim=0 105 0 110, clip,
            offset=0 105,
            pagecommand={}]{dummy}
\end{example}

%As a default \verb|\includepdf| uses the option 
%\verb|pagecommand={\thispagestyle{empty}}|.
%To remove this (if you want to use the headers and
%footers of your document) you can say:
%\verb|pagecommand={}|

The option
\verb|pagecommand={\thispagestyle{empty}}|
is set by default.
To remove this (if you want to use the headers and
footers of your document) you can set:
\verb|pagecommand={}|

\includepdf[page=3,
            trim=0 105 0 110, clip,
            offset=0 105,
            pagecommand={}]{dummy}

\end{document}
%</example>
%\fi
\endinput
